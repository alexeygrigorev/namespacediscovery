\documentclass[]{article}
\usepackage{draftwatermark}
%opening
\title{Discovering Namespaces in Mathematical Notation}
\author{Alexey, Moritz, Sergey, \dots , Volker}

\begin{document}

\maketitle

\begin{abstract}
While modern programming languages use name-spaces for means of modularity and expandability mathematical notation has no such concept.
However, most scientific communities have established standard notation for mathematics relevant to their fields.
We claim that the sharing of notation corresponds to the taxonomic distance of the research fields.
Nowadays, digital communication plays a significant role in the transportation of mathematical concepts and ideas.
We see advantages in using name-spaces for mathematical notation to reduce ambiguity and increase the widespread of ideas across community boarders.
In this paper we extract identifier definition tuples from Wikipedia, and map them to classification sachems for mathematics and physics. Thereby, we get a hierarchy of identifier definition tuples for pairs.
%TODO: Discuss, if we want to use arXiv articles that contain category information.
\end{abstract}

\section{The vision of namespaced Mathematics}
Motivation and introduction goes here
\section{Background and Related Works}
Keep this section brief. Find additional references from other research fields like linguistics?
\section{The Machinery for Namespace Discovery}
Brief discription of the tools
\section{The Wikipedia Case Study}
Result of the Master Thesis
Fuse bilingual result.
\section{Name spacing the ArXiv}
Needs to be done. In general this follows the same pattern.

Discuss with Deyan Ginev to get the full HTML5 corpus.

See if clustering methods can cope with the data volume? 100M formula probably 1 billion identifiers.

\section{Learning outcome: Namespaced Identifiers}
Write evaluation
\section{A long road ahead}
Write future work
\end{document}
