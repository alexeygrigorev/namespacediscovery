\section*{Abstract}

In computer science, a \emph{namespace} refers to a collection of
terms that are grouped together because they share functionality or
purpose, typically for providing modularity and resolving name conflicts
\cite{duval2002metadata}.
Programmers recognize the importance of namespaces: with namespaces, it is 
possible to add the structure and hierarchical organization to the source code
\cite{mcarthur2008php6}.

In mathematics, the same identifier may be used in different areas
but denote different things, and this
ambiguity is confusing for people who are learning a particular 
area. It would be helpful for them to have a clear way of organizing 
identifiers into a hierarchical structure, in the same way the source 
code is organized in programming languages.
%TODO: Describe the consequences for Mathematicians and especially people
% coming from other fields that are confused by ambiguity describe the idea
% to transfer the NS concept originating from CS to STEM.
In this thesis we extend the notion of namespaces to mathematical formulae.
By introducing namespaces to mathematics not only the ambiguity can be resolved, 
but also the mathematical knowledge can be systematized,
thus helping scientists to comprehend the notation in new areas quicker. 

It is not feasible to construct identifier namespaces manually:
for a data set with 300\,000 identifiers, assuming that the speed of 
assigning an identifier to its namespace is 30 seconds,
% and also that the person performing the task is knowledgeable , 
it would take approximately 3.4 months to complete the task.
% TODO: Describe how many formulae are out there. Estimate the effort it would be to
% manually update all identifiers "it is not feasible" is not justified. Get rid
% of all unjustified statements in your thesis and the quality will increase a lot.
In this work we propose an automatic method for namespace discovery, and our goal 
is to discover namespaces in mathematical identifiers given a collection of scientific 
documents.

We recognize that the problem of namespace discovery can be seen as a 
clustering problem.
In addition, we argue that document representation in terms of identifiers is similar to the 
traditional way of representing documents in the Vector Space Model. 
By comparing both ways of representing documents, we develop a method based 
on techniques from document clustering: first, we apply cluster analysis algorithms 
to discover document clusters, and then, we retain only pure ``namespace-defining'' clusters.
The purity of clusters is calculated based on the category information, and a 
cluster is \emph{namespace-defining}, if it its documents are about the same domain. 

%By comparing these %TODO: which?
%approaches we develop a method
%TODO: This is too generic
%for automatic namespace discovery based on techniques from
%document clustering.
%Not all discovered clusters are equally good, and
%We apply additional filtering to select namespace-defining clusters based on
%document categories. %TODO: I don't understand that. How do we EVALUATE?

To cluster documents, we use established cluster analysis algorithms like
$K$-Means, DBSCAN and Latent Semantic Analysis, and we look for an algorithm 
that discovers as many namespace-defining clusters as possible. 
These techniques are evaluated on English Wikipedia, and additionally, we use Russian
Wikipedia to compare the performance of our approach on a different dataset. 
We observe that the results are consistent across the languages. However there is some 
difference in the distribution of extracted namespaces: the most common 
namespaces discovered from English Wikipedia are different from the namespaces 
from Russian Wikipedia, and this is expected because the datasets are different. 
%TODO: Describe the differences and why they were expected
% which makes the direct comparison difficult.

To our knowledge, the problem of discovering namespaces in mathematical notation
has not been addressed before, and there are no established techniques that 
work best. In this work we had to progress through trial and error, and 
due to limited amount of time, it was not possible to evaluate all possible 
approaches.

However, the obtained results indicate that the identifier namespace discovery is
possible, although we expected to discover more namespaces. We also outline many 
ways to improve the results further.

%TODO: This is not very convincing. You should consider that you were the first
% to work on this difficult problem and for your master thesis your time was
% restricted to nominal 900 hours which usually can be multiplied with a factor
% of 1.5 to estimate the time people expect you to have spent on your thesis.
