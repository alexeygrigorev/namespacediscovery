\section*{Abstract}


In computer science, a \emph{namespace} refers to a collection of
terms that are managed together because they share functionality or
purpose, typically for providing modularity and resolving name conflicts.

In this thesis we extend the notion of namespaces to mathematical formulae.
In mathematics, the same identifier may be used in different areas
but denote different things. By introducing namespaces to
formulae, this ambiguity can be resolved. 

It is not feasible to construct identifier namespaces manually, and 
therefore we propose a method for namespace discovery given a collection
of scientific documents. We argue that document representation in terms 
of identifiers is similar to the traditional way of representing documents
in the Vector Space Model. By comparing these approaches we
develop a method for automatic namespace discovery based on techniques from
document clustering. Not all discovered clusters are equally good, and we apply 
additional filtering to select namespace-defining clusters based on 
document categories.

To cluster documents we use established cluster analysis algorithms like
$K$-Means, DBSCAN and Latent Semantic Analysis, and these techniques
are evaluated on English Wikipedia. Additionally, we use Russian
Wikipedia to compare the performance of our approach on a different dataset, and
the results are consistent across the languages. However there is some difference
in the distribution of extracted namespaces, which makes the direct 
comparison difficult. 

The obtained results indicate that the identifier namespace discovery is
possible, although we expected to discover more namespaces. We see many ways 
to improve the results further.
