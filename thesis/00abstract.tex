\section*{Abstract}

In Computer Science, namespaces help to structure source code and
organize it into hierarchies. Initially, the concept of namespaces did not
exist for programming languages, and programmers had to manage the source code % namin conventions?
 themselves
to ensure there were  no name conflicts. However, nowadays, namespaces are
adopted by the majority of modern programming languages.

The concept of namespaces is beneficial for mathematics as well:
In mathematics, short one-symbol identifiers are very common,
and the meaning of these identifiers is hard to understand immediately.
By introducing namespaces to mathematics, we will be able to organize mathematical
identifiers. Also, mathematicians will also benefit from a hierarchical organization of
knowledge in the same way programmers do. In addition, the structure
will make it easier to understand the meaning of each identifier in a document.

In this thesis, we look at the problem of assigning each identifier of a document
to a namespace. At the moment, there does not exist a special dataset where
all identifiers are grouped %is grouping enough?
 to namespaces, and therefore we need to create
such a dataset ourselves.

Namespaces are hard to prepare manually: building them requires a lot of time and effort.
However, it can be done automatically, and we propose
a method for automatic namespace discovery from a collection of documents.

To do that, we need to find groups of documents that use identifiers in the same way.
This can be done with cluster analysis methods.
We argue that documents can be represented %what does that represent?
by the identifiers they contain, and this
approach is similar to representing textual information in the Vector Space Model.
Because of this, we can apply traditional document clustering techniques for namespace
discovery.

To evaluate the results, we use the category information, and look for pure
``namespace-defining'' clusters: clusters where all documents are from the same category.
In the experiments, we look for algorithms that discover as many namespace-defining
clusters as possible.

Because the problem is new, there is no gold standard dataset, and it is
hard to evaluate the performance of our method. To overcome it, we first
use Java source code as a dataset for our experiments, since it contains the namespace
information. We verify that our method can partially recover namespaces
from source code using  only information about identifiers. %recover namespaces? what's about ns def clusters here?

The algorithms are evaluated on the English Wikipedia, and the proposed method
can extract namespaces on a variety of topics. After extraction, the namespaces
are organized into a hierarchical structure by using existing classification schemes
such as MSC, PACS and ACM.
We also apply it to the Russian Wikipedia, and the results are consistent across
the languages.

To our knowledge, the problem of introducing namespaces to mathematics has not
been studied before, and prior to our work there has been no dataset where identifiers
are grouped into namespaces. Thus, our result is not only a good start,
but also a good indicator that automatic namespace discovery is possible. 


\section*{Zusammenfassung}
Namespaces in Informatik helfen um den Quellcode zu organisieren.
Namespaces sind auch f\"ur Mathematik vorteilhaft. Durch die Einf�hrung von Namespaces in der Mathematik, kann mann mathematische Identifikatoren in einer hierarchy organisieren.

In dieser Masterarbeit betrachten wir das Problem der Zuordnung jeder Identifikator eines Dokuments zu einem Namespaces. Im Moment gibt es keine spezielle Datenmenge in der alle Mathematische Identifikatoren zu Namespaces zugeordnet sind. Deshalb m\"ussen wir eine Datenmenge selbst erstellen. Wir schlagen eine automatisierte methode vor f\"ur die Entdeckung von der Zuordnung von Namespaces zu Identifikatoren aus einer Sammlung von Dokumenten.

Um dies zu erreichen, m\"ussen wir Gruppen von Dokumenten finden, die Identifikatoren in der gleichen Weise verwenden.
Das kann mit Clustering-Algorithmen durchgef�hrt werden. Wir schlagen vor dass Dokumente repr\"a\-sentiert werden k\"onnen durch die Identifikatoren die sie enthalten. Dieser Ansatz ist \"ahnlich zu der Abbildung von Textinformationen in dem Vektorraummodell.
Aus diesem Grund k\"onnen wir traditionelle Dokument Clustering-Algorithmen f\"ur Namespace Entdeckung gelten.

Um die Ergebnisse zu bewerten, verwenden wir die Kategorieinformationen, und suchen wir nach reinen ``namespace-definierenden'' Cluster: Cluster, in dem alle Dokumente zu der gleichen Kategorie geh\"oren.
In die Experimenten suchen wir nach Algorithmen, die so viele reine Cluster wie m�glich entdecken.

Die Algorithmen sind auf der Englischen Wikipedia ausgewertet.
Unsere Methode kann Namespaces auf einer Vielzahl von Themen extrahieren.
Nach der Extraktion, organisieren wir die Namespaces in einer Hierarchie.
Wir setzen unsere Methode auch auf der Russischen Wikipedia und finden \"anlische Ergebnisse.
Unser Ergebnis zeigt, dass die automatische Namespaces Entdeckung m\"oglich ist.
%This German summary is not acceptable.