\section*{Abstract}

% Motivation 
In Computer Science, namespaces help to structure source code and
organize it into hierarchies. Initially, the concept of a namespace did not 
exist for programming languages, and programmers had to manage the source code
such that there was no name conflicts. However, nowadays, the majority 
of modern programming languages have adopted this concept and have namespaces. 

The concept of namespaces is beneficial for mathematics as well: 
In mathematics, short one-symbol identifiers are very common, 
and the meaning of these identifiers is hard to understand immediately. 
If we transfer the concept of namespaces to mathematics for organizing mathematical identifiers, 
mathematician will also benefit from a hierarchical organization of 
knowledge in the same way as programmers. In addition, this structure 
will make it easier to understand the meaning of each identifier in a document.


% Goal 
In this thesis, we look at the problem of assigning each identifier from a document 
to a namespace. At the moment, there does not exist a special dataset where 
all identifiers are grouped to namespaces, and therefore we need to create 
such a dataset ourselves.


% Method + evaluation
Namespaces are hard to prepare manually: it requires a lot of time and effort. 
However, it can be done automatically, and we propose 
a method for automatic namespace discovery from a collection of documents. 

First, we locate groups where documents use identifiers in the same way, and this 
is a cluster analysis problem.
We argue that documents can be represented by the identifiers they contain, and this 
approach is similar to representing textual information in Vector Space Model.
Because of this, we apply traditional document clustering techniques for namespace 
discovery.

Then, to evaluate the results, we use the category information, and look for pure 
``namespace-defining'' clusters: clusters where all documents are from the same category. 
In the experiments, we look for algorithms that discover as many namespace-defining 
clusters as possible.

The algorithms are evaluated on English Wikipedia, and the proposed method 
can extract namespaces on a variety of topics. 
We also apply it to Russian Wikipedia, and the results are consistent across 
the languages.

% Result
To our knowledge, the problem of introducing namespaces to mathematics has not
been studied before, and prior to our work there has been no dataset where identifiers 
are grouped into namespaces. Thus, our result is not only a good start, 
but it also a good indicator that automatic namespace discovery is possible.