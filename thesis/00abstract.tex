\section*{Abstract}


In computer science, a \emph{namespace} refers to a collection of
terms that are managed together because they share functionality or
purpose, typically for providing modularity and resolving name conflicts.
%TODO: Where does this definition come from. I would prefer to argue that
% NS are heavily used and greatly appreciated by programmer to ...

In mathematics, the same identifier may be used in different areas
but denote different things. 
%TODO: Describe the consequences for Mathematicians and especially people
% coming from other fields that are confused by ambiguity describe the idea
% to transfer the NS concept originating from CS to STEM.
In this thesis, we extend the notion of namespaces to mathematical formulae.
By introducing namespaces to
formulae, this ambiguity can be resolved. 

It is not feasible to construct identifier namespaces manually,
% TODO: Describe how many formulae are out there. Estimate the effort it would be to
% manually update all identifiers "it is not feasible" is not justified. Get rid
% of all unjustified statements in your thesis and the quality will increase a lot.
and 
therefore we propose a method for namespace discovery given a collection
of scientific documents.
%We argue that 
Document representation in terms 
of identifiers is similar to the traditional way of representing documents
in the Vector Space Model. 
By comparing these %TODO: which?
approaches we develop a method
%TODO: This is too generic
for automatic namespace discovery based on techniques from
document clustering. 
%Not all discovered clusters are equally good, and 
We apply additional filtering to select namespace-defining clusters based on 
document categories. %TODO: I don't understand that. How do we EVALUATE?

To cluster documents we, use established cluster analysis algorithms like
$K$-Means, DBSCAN and Latent Semantic Analysis, and these techniques
are evaluated on English Wikipedia. Additionally, we use Russian
Wikipedia to compare the performance of our approach on a different dataset, and
the results are consistent across the languages. However there is some difference
%TODO: Describe the differences and why they were expected
in the distribution of extracted namespaces, which makes the direct 
comparison difficult. 

The obtained results indicate that the identifier namespace discovery is
possible, although we expected to discover more namespaces. We see many ways 
to improve the results further.
%TODO: This is not very convincing. You should consider that you were the first
% to work on this difficult problem and for your master thesis your time was
% restricted to nominal 900 hours which usually can be multiplied with a factor
% of 1.5 to estimate the time people expect you to have spent on your thesis.
