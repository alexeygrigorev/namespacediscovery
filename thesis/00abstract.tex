\section*{Abstract}


In computer science, a \emph{namespace} refers to a collection of 
terms that are managed together because they share functionality or 
purpose, typically for providing modularity and resolving name conflicts. 
For example, XML uses namespaces to prefix element names to ensure uniqueness 
and remove ambiguity between them, and the Java programming language uses 
packages to organize identifiers into namespaces for modularity.

In this thesis we extend the notion of namespaces to mathematical formulae.
In mathematics, the same identifier may be used in different areas 
but denote different things: For example, ``$E$'' may refer to ``energy'', 
``expected value'' or ``elimination matrix'', depending on the domain of the article 
where this identifier is used. By introducing namespaces to 
formulae, this ambiguity can be resolved. If we follow the XML approach
and prepend namespace name to the identifier, e.g. ``physics.$E$'', then it 
will give additional context and make it clear that ``physics.$E$'' means ``energy''
rather than ``expected value''.

There is correlation between  the identifiers used in a document and
the namespace of its identifiers, and we exploit it to discover namespaces.
We argue that document representation in terms of identifiers is 
similar to the traditional way of representing documents as a Bag of Words
in the Vector Space Model. By comparing these approaches we 
develop a method for automatic namespace discovery based on techniques from 
document clustering. 

To cluster documents we use established cluster analysis algorithms like 
$K$-Means, DBSCAN and Latent Semantic Analysis, and these techniques 
are evaluated on English Wikipedia. Additionally, we also use Russian 
Wikipedia to compare the obtained results, and try the same set of 
techniques to extract namespaces from Java source code.
