\section{Introduction}

In computer science, a \emph{namespace} refers to a collection of terms that are managed 
together because they share functionality or purpose, typically for providing modularity 
and resolving name conflicts \cite{duval2002metadata}. For example, XML uses namespaces 
to prefix element names to ensure uniqueness and remove ambiguity between
them~\cite{}, and the Java programming language uses packages to organize
identifiers into namespaces for modularity~\cite{gosling2014java}.

In this thesis we will extend the notion of namespaces to mathematical formulae.


In logic, a \emph{formula} is defined recursively, and, in essence, 
it is a collection of variables, functions and other formulas, and formally the symbols for
the variables and functions can be chosen arbitrarily \cite{barwise2000language}. 
However, in contrast to first order logic, in this work we are interested in the symbols 
in formulae and in mathematical notations that are used by different research communities.
For example, in physics it is common to write the energy-mass relation as $E=mc^2$ rather 
than $x=yz^2$.

Identifier 


However, the same identifier may be used in different areas but denote different things: 
For example, $E$ may refer to ``energy", ``expected value" or ``elimination matrix", depending on
the domain of the article. Thus, we can note that these identifiers form namespaces, and we 
refer to such namespaces as \emph{identifier namespaces}, and to the process of discovering
identifier namespaces as \emph{namespace disambiguation}.


In this thesis we compare different approaches for namespace disambiguation. 
The first approach is to assume that there is a strong correlation between identifiers in 
a document and the namespace of the document, and this correlation can be exploited to 
categorize documents and thus discover namespaces. For example, if we observe a document with 
two identifiers $E$, assigned to ``energy'', and $m$, assigned to ``mass'', then it is more 
likely that the document belongs to the ``physics'' namespace rather than to ``statistics''. 
To use it, we need to map identifiers to their definitions, and this can be done by extracting 
the definitions from the text that surrounds the formula \cite{pagael2014mlp}. Other approaches 
are based on the text of the documents, rather on the formulae \cite{sebastiani2002machine}, 
but nonetheless we believe that there is a correlation between the textual content of a document 
and the namespace of its identifiers.






\section{Goals}

The main objective of this study is to discover identifier namespace in mathematical formulae.
We aim to find \emph{meaningful} namespaces, in the sense that they can be related to a real-world area of knowledge, such as physics, linear algebra or statistics.

Once such namespaces are found, they can give good categorization of scientific documents based on formulae and notation used in them.

We believe that this may facilitate better user experience: for instance, it will allow users to navigate easily between documents of the same category and see in which other documents a particular identifier is used, how it is used, how it is derived, etc. Additionally, it may give a way to avoid ambiguity. If we follow the XML approach \cite{xmlnamespaces} and prepend namespace to the identifier, e.g. ``physics.$E$'', then it will give additional context and make it clear that ``physics.$E$'' means ``energy" rather than ``expected value".

We also expect that using namespaces is beneficial for relating identifiers to definitions. Thus, as an application of namespaces, we would like to be able to use them for better definition extraction. It may help to overcome some of the current problems in this area, for example, the problem of \emph{dangling identifiers} \cite{pagael2014mlp} - identifiers that are used in formulae but never defined in the document. Such identifiers may be defined in other documents that share the same namespace, and thus we can take the definition from the namespace and assign it to the dangling identifier.

