\section{Introduction}

\subsection{Motivation}

In computer science, a \emph{namespace} refers to a collection of terms
that are grouped because they share functionality or purpose,
typically for providing modularity and resolving name conflicts \cite{duval2002metadata}.
For example, XML uses namespaces to prefix element names to ensure uniqueness and remove ambiguity between them \cite{xmlnamespaces}, and the Java programming language uses packages to organize identifiers into namespaces for modularity \cite{gosling2014java}.

In this thesis we extend the notion of namespaces to mathematical formulae. 
In mathematics, there exists a special system of choosing identifers, and it is called 
\emph{mathematical notation} \cite{wikinotation}. Because of the notation, 
when people write ``$E=mc^2$'', the meaning of this expression is recognized among scientists. 
However, the same identifier may be used in different areas, but denote 
different things: For example, ``$E$\,'' may refer to ``energy'', ``expected value'' or 
``elimination matrix'', depending on the domain of the article.
We can compare this problem with the problem of name collision in computer science 
and introduce namespaces of identifiers in mathematical notation to overcome it. 

In this work we aim to discover namespaces of identifiers in mathematical notation. 
However, the notation only exists in the documents where it is used, and it does 
not exist in isolation. 
It means that the identifer namespaces should be discovered from the documents 
with mathematical formulae. 
Therefore, the goal of this work is to \textbf{automatically discover a set of identifier
namespaces given a collection of documents}.

% \textbf{TODO: Should I describe here that we want to use the document clustering
% techniques? or it goes to the chap3?}

We expect the namespaces to be meaningful, in the sense that they can be related to real-world areas of knowledge, such as physics, linear algebra or statistics.

Once such namespaces are found, they can give good categorization of scientific 
documents based on formulas and notation used in them. We believe that this may 
facilitate better user experience: when learning a new area it will help the users 
familiarize with the notation faster.
Additionally, it may also help to locate the usages of a particular identifier and 
refer to other documents where the identifier is used. 

Namespaces also give a way to avoid ambiguity. If we refer to an identifier from a 
particular namespace, then it is clear what the semantic meaning of this 
identifier. For example, if we say that ``$E$\,'' belongs to a namespaces 
about physics, it gives additional context and makes it clear that ``$E$\,''
means ``energy'' rather than ``expected value''.

Finally, using namespaces is beneficial for relating identifiers to 
definitions. Thus, as an application of namespaces, we can use them for better 
definition extraction. It will help to overcome some of the current problems in 
this area, for example, the problem of dangling identifiers~-- identifiers that are 
used in  formulas but never defined in the document \cite{pagael2014mlp}. 
Such identifiers may be defined in other documents that share the same namespace, 
and thus we can take the definition from the namespace and assign it to the dangling identifier.




\subsection{Thesis Outline}

The thesis is organizes as follows: 

\begin{description}
\item[Chapter \ref{sec:background} -- \nameref{sec:background}] \hfill \\
  In this chapter we do a survey of the related work. We discuss how namespaces are
  used in Computer Science. Secondly, we review how definitions for identifiers 
  used in mathematical formulae can be extracted from the natural language text
  around the formulae. Finally, we review the Vector Space Model~-- a way of 
  transforming texts to vectors, and then discuss how these vectors can be 
  clustered. 

\item[Chapter \ref{sec:namespace-discovery-chap} -- \nameref{sec:namespace-discovery-chap}] \hfill \\
  In chapter \ref{sec:namespace-discovery-chap} we introduce the problem of namespaces 
  in mathematical notation, discuss its similarities with namespaces in Computer Science,
  and propose an approach to namespace discovery by using document clustering techniques. 
  We also extend the Vector Space Model to represent identifiers and suggest several 
  ways to incorporate definition information to the vector space. 

\item[Chapter \ref{sec:implementation} -- \nameref{sec:implementation}] \hfill \\
  Chapter \ref{sec:implementation} describes how the proposed approach is implemented. 
  It includes the description of the data sets for our experiments, and the details of 
  implementation of definition extraction and document cluster analysis algorithms. 
  Additionally, we propose a way of converting document clusters to namespaces.

\item[Chapter \ref{sec:evaluation} -- \nameref{sec:evaluation}] \hfill \\
  In chapter \ref{sec:evaluation} we describe the parameter selection procedure,
  and we present the results of the best performing method. Once the clusters are 
  discovered, they are mapped to a hierarchy, and we summarize our findings 
  by analyzing the most frequent namespaces and most frequent identifier-definition
  relations in these namespaces.

\item[Chapter \ref{sec:conclusions} -- \nameref{sec:conclusions}] \hfill \\
  Chapter \ref{sec:conclusions} summarizes the findings.

\item[Chapter \ref{sec:future-work} -- \nameref{sec:future-work}] \hfill \\
  Finally, in chapter \ref{sec:future-work} we discuss the possible areas of improvements.
  We conclude this chapter by identifying the questions that are not resolved and 
  present challenges for future research on identifier namespace discovery.

\end{description}