\section{Introduction}

\subsection{Namespaces in Computer Science}

In computer science, a \emph{namespace} refers to a collection of terms that are managed
together because they share functionality or purpose, typically for providing modularity
and resolving name conflicts \cite{duval2002metadata}.


XML (eXtensible Markup Language) is a framework for defining markup languages.
XML lets users define a set of tags to represent information in some
specific domain \cite{moller2006introduction}. For example, XHMTL is an XML
language for hypertext markup and MathML is a language for describing mathematical
notation.

However, different XML languages may use the same names for elements and attributes.
For example, consider two XML languages: XHTML for specifying the layout of web
pages, and some XML language for describing furniture. Both these languages have
the \verb|<table>| elements there, in XHTML table is used to present some data in
a tabular form, while the second one uses it to describe a particular piece of
furniture in the database.

The \verb|<table>| elements have very different semantics in these languages
and there should be a way to distinguish between these two elements.
In XML this problem is solved with XML namespaces~\cite{xmlnamespaces}:
the namespaces are used to ensure the uniqueness of attributes and resolve ambiguity.
It is done by binding a short namespace alias with some uniquely defined URI
(Unified Resource Identifier), and then appending the alias to
all attribute names that come from this namespace. In the example above,
we can bind an alias \verb|h| with XHTML's URI \url{http://www.w3.org/TR/xhtml1}
and then use \verb|<h:table>| to refer to XHTML's table. Likewise,
in the furniture database language the element names can be prepended
with a prefix \verb|d|, where \verb|d| is bound to some URI, e.g.
\url{http://www.furniture.de/2015/db}.


% Ralf's comment: ``namespaces" were deeply studied mainly in
% the field of distributed systems/ middleware (like DCE, CORBA, etc.) in the middle
% of the 80s/ 90s ...


The namespaces are also used in programming languages for organizing
variables, procedures and other identifiers into groups and
for resolving name collisions. In programming languages without
namespaces the programmers have to take special care to avoid
naming conflicts. For example, in the PHP programming language
prior to version 5.3 \cite{mcarthur2008php6} there is no notion of namespace, and
the namespaces have to be emulated to ensure that the names
are unique, and this often results in long names like
\verb|Zend_Search_Lucene_Analysis_Analyzer|\footnote{\url{http://framework.zend.com/apidoc/1.7/Zend_Search_Lucene/Analysis/Zend_Search_Lucene_Analysis_Analyzer.html}}.


Other programming languages have the notion of namespaces built in
from the very first versions. For example, the Java programming
language~\cite{gosling2014java} uses packages to organize identifiers into
namespaces. In Java, packages solve the problem of ambiguity. For example,
in the standard Java API there are two classes with the name \texttt{Date}:
one in the package \texttt{java.util} and another in the package \texttt{java.sql}.
To be able to distinguish between them, the classes are referred by their
\emph{fully qualified name}: an unambiguous name that uniquely specifies the class
by combining the package name with the class name. Thus, to refer to a particular
\texttt{Date} class in Java  \texttt{java.util.Date} or  \texttt{java.sql.Date}
should be used. 

It is not always convenient to use the fully qualified name in the code to
refer to some class from another package. Therefore in Java it is possible to
\emph{import} the class by using the import statement which associates
a short name alias with its fully qualified name.
For example, to refer to \texttt{java.sql.Date} it is possible to import
it by using \texttt{import java.sql.Date} and then refer to it by the alias
\texttt{Date} in the class \cite{gosling2014java}.


Although there is no strict requirement to organize the classes into
well defined groups, it is a good software design practice to put
related objects into the same namespace and by doing this achieve
better modularity. There are design principles that tell software engineers
how to best organize the source code: classes in a well designed system
should be grouped in such a way that namespaces
exhibit low \emph{coupling} and high \emph{cohesion}~\cite{larman2005applying}.
Coupling describes the degree of dependence between namespaces, and
low coupling means that the interaction between classes of different
namespaces should be as low as possible. Cohesion, on the other hand,
refers to the dependence within the classes of the same namespace,
and the high cohesion principle says that the related classes
should all be put together in the same namespace.

% TODO What's the significance?


\subsection{Namespaces in Mathematical Notation}

Informally, a mathematical formula is a rule that shows the relationship
between different variables. For example,
$x_{1,2} = \frac{-b \pm \sqrt{b^2 - 4 a c}}{2 a}$ is a formula for solving
a quadratic equation $a x^2 + b x + c = 0$.


To give a more formal definition of formula, we first need to define
a first-order language that contains primitive symbols such as
(1) parentheses, brackets and other boundary symbols
(2) \emph{constants} ($1$, $2$, $3$, ...) and variables ($x$, $y$, ...)
(3) \emph{functions} ($+$, $\cdot$, ...) and
(4) \emph{predicates}, e.g. binary relation symbols (``$=$'', ``$<$'', ``$\geqslant$'', ...)
\cite{barwise2000language}.

In this language, \emph{constants} are symbols with pre-defined meaning
from some alphabet and variables are symbols that can be assigned
a value from this alphabet.
Any symbol can be a variable, for example, $x$, $y$, $\mathbf w$, or it can
be a symbol with subscripts, for example, $x_1, x_2, \ ...$ or even $w_\text{slope}$.

A \emph{well-formed term} $t$ (or just \emph{term}) in this language is defined as
$$t \equiv c \mid x \mid f(t_1, t_2, \ ... \ , t_n) \ ,$$ which means that the term
$t$ can be a constant, a variable or an $n$-ary function
$f(t_1, t_2, \ ... \ , t_n)$. An  \emph{$n$-ary function} is an
function that takes $n$ terms $t_1, t_2, \ ... \ , t_n$ and produces
some new term. An \emph{$n$-ary predicate} (or an \emph{$n$-ary relation symbol})
is typically a boolean-valued function that can be evaluated to \texttt{True}
or \texttt{False} depending on the values it gets.

Then an \emph{atomic well-formed formula} (or just \emph{formula}) in this language
is a $n$-ary predicate with $n$ terms evaluated to \texttt{True} \cite{barwise2000language}.

For example, $x_{1,2} = \frac{-b \pm \sqrt{b^2 - 4 a c}}{2 a}$ is a formula,
because it represents an equation that always holds true for a quadratic equation
$a x^2 + b x + c = 0$. The equality symbol ``$=$'' is a predicate that shows
the relationship between variables $x_1, x_2$ and variables $a, b, c$.

In logic we can use any symbol for variables without changing the meaning of the formula.
For example, the energy-mass equivalence relation $E = mc^2$ can be written
as $x = y z^2$ and it still will still hold true and remain a valid formula.
However, there are research communities in mathematics that have developed
a special system of naming these variables, and these naming systems are
called \emph{mathematical notations} \cite{wikinotation}. For each symbol in a formula,
the notation assigns a precise semantic meaning. Therefore, because of the
notation, in Physics it is more common to write $E = mc^2$ rather
than $x = y z^2$, because the notation assigns unambiguous meaning to the symbols
``$E$'', ``$m$'' and ``$c$'', and the meaning of these symbols is recognized
among physicists.

% TODO: ``notation'' taken from wiki - need to find a better source
% https://en.wikipedia.org/wiki/Mathematical_notation
% maybe Wolska's thesis?

However the notations may conflict. For example, while it is common to use
symbol $E$ to denote ``Energy'' in Physics, it also is used in Probability and
Statistics to denote ``Expected Value'', or in Linear Algebra to denote
``Elimination Matrix''.
We can compare the conflict of notations with the name collision problem in namespaces,
and solve it by extending the notion of namespaces to mathematical notation.

Thus, let us define a \emph{notation} $\mathcal N$ as a set of pairs $\{ (i, s) \}$,
where $i$ is a symbol and $s$ is its semantic meaning,
such that for any pair $(i, s) \in \mathcal N$ there does not exist a pair
$(i', s') \in \mathcal N$ with $i = i'$.
Let us call $i$  \emph{identifier} and $s$ \emph{definition}, and
then $(i, s)$ is an \emph{identifier-definition pair}.
Two notations $\mathcal N_1$ and $\mathcal N_2$
\emph{conflict}, if there exists a pair $(i_1, s_1) \in \mathcal N_1$ and a pair
$(i_2, s_2) \in \mathcal N_2$ such that $i_1 = i_2$ and $s_1 \ne s_2$.

Then we can define \emph{namespace} as a named notation, i.e. a pair
$(\text{uid}, \mathcal N)$ where $\mathcal N$ is a notation and ``uid'' is
a some string that uniquely identifies the notation. For convenience, in this
work we can use the Java syntax to refer to specific entries of a namespace. If
$(\text{name}, \mathcal N)$ is a namespace and $i$ is an identifier such that
$(i, s) \in \mathcal N$ for some $s$, then ``name.$i$'' is a
\emph{fully qualified name} of the identifier $i$ that relates it to the
definition $s$. For example,
given a namespace $\big( \text{``Physics''},$ $\{ (E, \text{``energy''}),$
$(m, \text{``mass''}),$ $(c, \text{``speed of light''}) \} \big)$,
``Physics.$E$'' refers to ``energy'' -- the definition of $E$ in the
namespace ``Physics''.

% TODO extend it to hierarchies ?

Analogously to namespaces in Computer Science, formally a mathematical namespace
can contain any set of identifier-definition pairs that satisfies the definition of
the namespace, but typically namespaces of mathematical notation
exhibit the same properties as well-designed Java packages: they
have low coupling and high cohesion, meaning that all definitions
come from the same area of mathematical knowledge and the definitions
from different notations do not intersect heavily.

Additionally, we can introduce a document-centric view on the mathematical
namespaces: suppose we have a collection of documents of $n$ documents
$\mathcal D = \{ d_1, d_2, \ ... \ , d_n \}$ and a set of $K$ namespaces
$\{(n_1, \mathcal N_1), (n_2, \mathcal N_2), \ ... \ ,$ $(n_K, \mathcal N_K) \}$.
A document $d_j$ can use a namespace $(n_k, \mathcal N_k)$ by \emph{importing}
identifiers from it. To import an identifier, the document uses an import statement
where the identifier $i$ is referred by its fully qualified name
defined by this namespace.





\textbf{TODO}: Continue, re-introduce low coupling and high cohesion in terms of
document collection.


However, in real-life scientific documents there are no import statements in the document
preamble, and therefore ...

The goal is to automatically discover the set of mathematical namespaces
given a collection of documents.

Manual labeling is time consuming

Want: approximation via ML algorithms

To do that need to be able to extract definition from text

% How to find a namespace? Can be constructed manually using existent category information
%(refer to that list of categories of scientific articles)

% But it's very time consuming. Our approach: use ML techniques to automatically discover
% the namespaces from corpus with mathematical formulas.

% To use it, we first need to map identifiers to their definitions, and this can be done by
% extracting the definitions from the text that surrounds the formula. We explore these methods
% in section \ref{section:definitionextraction}.

\subsection{Namespaces in Mathematical Notation} \label{sec:namespaces}


An \emph{identifier namespace} is a coherent structure where
each identifier is used only once and has a unique definition.

How to find a namespace? Namespaces can be constructed by
manually labeling each identifier/definition pair with appropriate
name. But this is very time consuming, and in this work we suggest
a different approach: use Machine Learning techniques for
discovering namespaces automatically from a collection of scientific
documents containing mathematical formulae.


Many modern programming languages use namespaces for modularity.
For example, in the Java programming language \cite{gosling2014java}
namespaces are called ``packages'' and
a class may refer to classes from other packages via the \texttt{import}
statement.

% Example?


Typically in a well-designed application, we can distinguish between
two types of application packages \cite{evans2004domain}:

\begin{itemize}
  \item \emph{type 1}: domain-specific packages that deal with one particular concept, and
  \item \emph{type 2}: packages that use many other packages of the first type
\end{itemize}


For example, we have an application \verb|org.company.app|
with several domain-specific packages: \verb|org.company.app.domain.user|
with classes related to users, \verb|org.company.app.domain.account|
with classes related to user accounts, and a system-related package
\verb|org.company.app.tools.auth| that deals with authentication and
authorization. Then we also have a package \verb|org.company.app.web.manage|,
which belongs to the type 2: it handles web requests
while relying on classes from \verb|user| and \verb|account| to
implement the business logic and on \verb|auth| for making sure the
requests are authorized.

We can observe that the type 1 packages are mostly self-contained
and not highly coupled between each other,
but type2 packages mostly use other packages of type 1: they
depend on them.

\ \\

We can extend this idea to scientific documents and identifier
namespaces. A document can be seen as a class that uses concepts defined
in other documents. Then the documents can be grouped such that
some groups are of \emph{type 1}: they define the namespaces. In some
sense the documents of type 1 are ``pure'' - they contain infromation
about closely related concepts and not highly coupled
with other document groups. But some documents are of \emph{type 2} and they
are not pure: they draw from different concepts.


This intuition allows us to have the following assumptions:

\begin{enumerate}
 \item documents are ``mixtures'' of namespaces: they take identifiers from several namespaces
 \item there are some documents are more ``pure'' than others: they either take identifiers exclusively from one namespace or from few very related namespaces
\end{enumerate}


With these assumptions we can refer to ``pure'' groups as
\emph{namespace defining} groups. These groups can be seen as
``type 1'' packages: they define namespaces that are used by other
``type 2'' document groups.


Additionally, we can assume that there is a strong correlation
between identifiers in a document and the namespace of the document,
and this correlation can be exploit to categorize
documents into groups.


% The goal of this work is to automatically discover identifier namespaces in
% mathematical notation. But mathematical notation does not exist in isolation
% and it is usually used in documents. Therefore we can approximate
% namespace discovery from notation by namespace discovery from scientific
% documents.


Thus by combining these assumptions we can conclude that it should
be approximate the process of namespace discovery by discovering
groups of namespace defining documents, and this can be done by
applying cluster analysis techniques to documents, represented
by identifiers they contain.

In the next section we will argue why we can use traditional
document clustering techniques and what are the characteristics
that texts and identifiers have in common.


\subsection{Discovering Namespaces with Document Cluster Analysis} \label{sec:clusters-namespaces}


We believe that cluster analysis techniques developed for
text documents should work for identifiers.

Let us consider the characteristics of text data:

There are many distinct words in natural language. For example,
if $\mathcal V$ is a set of all possible words, then usually $|\mathcal V| \approx 10^5$,
but each individual document may contain only 500 distinct words, or
sometimes even less if we consider sentences or small documents
(e.g. tweets)

number of words across different document may wary a lot

 word distributions follow Power Laws (e.g. Zipf's law)


The identifiers have the same properties!

\ \\


Natural languages suffer from lexical problems of variability and ambiguity,
and the two main problems are synonymy and polysemy \cite{deerwester1990indexing}
\cite{gliozzo2009semantic}:

\begin{itemize}
\itemsep1pt\parskip0pt\parsep0pt
  \item two words are \emph{synonymous} if they have the same meaning
        (for example, ``word'' and ``term'' are synonyms),
  \item a word is \emph{polysemous} is it can have multiple meanings
        (for example, ``trunk'' can refer to a part of elephant or a part of a car).
\end{itemize}


We can note that identifiers have the same problems. For example,
in Information Theory, the Shannon Entropy is usually denoted by
``$H$'', but sometimes it is also denoted by ``$I$'' or by ``$S$'',
thus these identifiers may be seen as synonyms.
Also, ``$E$'' can stand both for ``Energy'' and ``Expected value'',
so ``$E$'' is polysemous.

These problems have been studied in Information Retrieval and
Natural Language Processing literature.
One possible solution for the polysemy problem is \emph{Word Sense Disambiguation}
\cite{jurafsky2000speech}: either replace a word with its sense
\cite{stokoe2003word} or append the sense to the word, for example
if the polysemous word is ``bank'' with meaning ``financial institution'',
then we replace it with ``bank\_finance''. The same idea can be used
for identifiers, for example ``$E$'' can be replaced with ``$E$\_energy''.
% We will expand this idea in the chapter~\ref{sec:ism}

Document clustering techniques usually use Vector Space Models
\cite{oikonomakou2005review} \cite{aggarwal2012survey} to represent documents.
We can define ``Identifier Spaces'' analogous to Vector Space Models.
We assume that Identifier Spaces exhibit the same characteristics as
traditional Vector Space Models, and thus

In the next section we review the Vector Space Model,
and then introduce the Identifier VSM in the chapter~\ref{sec:ism}

Then we can apply cluster analysis techniques to document-identifier matrices.



% \subsection{Outcomes}

% maybe there should be a better name for the section?

% \textbf{TODO:} taken form the proposal.
% The main idea here to describe what can happen if we solve the problem of
% namespace discovery in notation.



\subsection{Thesis Outline}


This work is organized as follows:
In chapter~\ref{sec:definitionextraction-top} we discuss how extract
definitions for identifiers in texts with mathematical formulae;
in chapter~\ref{sec:namespaces-top} the approaces to namespace
extraction and we argue why the cluster analysis can be used for that;
in chapter~\ref{sec:doc-clustering} we review cluster analysis methods
and in chapter~\ref{sec:latent-semantics} we discuss how to extract
latent information from data using different matrix factorization
techniques. Finally, we describe how the techniques are implemented
(chapter~\ref{sec:implementation}) and evaluated (chapter~\ref{sec:evaluation}).

