\section{Mathematical Definition Extraction} \label{sec:definitionextraction-top}

Mathematical expressions are hard to understand without the natural language description,
therefore we want to extract identifiers from mathematical expressions
and then find their definitions from the surrounding text.

For example, given the sentence ``The relation between energy and mass is
described by  the mass-energy equivalence formula $E = mc^2$,
where $E$ is energy, $m$ is mass and $c$ is the speed of light''
the goal is to extract the following identifier-definition relations:

\begin{itemize}
\itemsep1pt\parskip0pt\parsep0pt
  \item ($E$, ``energy'')
  \item ($m$, ``mass'')
  \item ($c$, ``the speed of light'')
\end{itemize}

Consider another example: ``Let $e$ be the base of natural logarithm''.
We would like to extract ($e$, ``the base of natural logarithm'').

Formally, a phrase that defines a mathematical expression consists of three parts \cite{kristianto2012extracting}:

\begin{itemize}
\itemsep1pt\parskip0pt\parsep0pt
  \item \emph{definiendum} is the term to be defined: it is a mathematical expression or identifier;
  \item \emph{definiens} is the definition itself: it is the word or phrase that defines the definiendum in a definition.
  \item \emph{definitor} is a relator verb that links definiendum and definiens.
\end{itemize}


In this work we are interested in the first two parts: \emph{definiendum} and
\emph{definiens}. Thus we define a \emph{relation} as a pair
(definiendum, definiens). For example, ($E$, ``energy'') is a relation where
$E$ is a definiendum, and ``energy'' is a definiens.


% \textbf{TODO}: An \emph{identifier} is a mathematical ...


We have the following assumption about defintion (i.e. deinities): 
Assumption: the definitions of mathematical expressions are always noun phrases

In general, a noun phrase can be

\begin{itemize}
\itemsep1pt\parskip0pt\parsep0pt
  \item a simple noun
  \item a compound noun (e.g. adjective + noun)
  \item a compound noun with a clause, prepositional phrase, etc
\end{itemize}



In this chapter we will discuss how the relations can be discovered automatically. 
The typical processing pipeline for definition extraction consists of 
the following steps \cite{kristianto2012extracting} \cite{pagael2014mlp}:

\begin{itemize}
\itemsep1pt\parskip0pt\parsep0pt
  \item Read corpus of documents in some markup language, e.g. Wiki Markup or Latex
  \item Translate all formulas in the documents into MathML
  \item Process MathML formulas
  \item Replace formulas with some placeholder
  \item Annotate text using Math-Aware POS Tagging
  \item Find relations in the text
\end{itemize}


Thus, this chapter is organized as follows: first we introduce the 
Mathematical Markup Language (MathML) in section~\ref{sec:mathml}, then discuss 
the Math-Aware POS Tagging procedure in section~\ref{sec:postagging} and 
finally review the extraction methods in section~\ref{sec:definition-extraction-methods}
and briefly discuss how the quality of extracted identifiers is evaluated
in section~\ref{sec:definition-evaluation}. 

%TODO:

% http://habrastorage.org/files/025/1f5/4fa/0251f54fa7a248faa6718839ee060b53.png
% MLP flow


\subsection{Formula Representation: MathML} \label{sec:mathml}

MathML \cite{mathml} stands for ``Mathematical Markup Language''
It is is a standard for mathematical
expressions defined by W3C that browsers should support to render math
formulas. There are two types of MathML: Presentation MathML, which describes
how mathematical expressions should be displayed, and Content MathML, which
focuses on the meaning of mathematical expressions. In this section, we will discuss
Presentation MathML .


A \emph{token} in MathML is an individual symbol, name or number. Tokens
are grouped together to form MathML expressions.

Tokens can be:

\begin{itemize}
\itemsep1pt\parskip0pt\parsep0pt
\item identifier, variable or function names
\item numbers
\item operators (including brackets - so called ``fences'')
\item text and whitespaces
\end{itemize}

A ``symbol'' is not necessarily one character: it could be a string such
as \texttt{<mi>sin</mi>} or \texttt{<mn>24</mn>}. 
In MathML they are treated as single tokens.


As in mathematics, MathML expressions are constructed recursively from
smaller expressions or single tokens. Complex expressions are created
with so-called ``layout'' constructor elements, while tokens are created
with token elements.

Let us consider an example. A mathematical expression $(a + b)^2$
can be represented in MathML as follows:

\begin{lstlisting}[language=XML,caption={TODO},label={}]
<math xmlns="http://www.w3.org/1998/Math/MathML">
  <msup>
    <mrow>
      <mo>(</mo>
      <mrow>
        <mi>a</mi>
        <mo>+</mo>
        <mi>b</mi>
      <mrow>
      <mo>)</mo>
    </mrow>
    <mn>2</mn>
  </msup>
</math>
\end{lstlisting}


It has the tree structure and recursive. If we take another mathematical
expression $\frac{3}{(a + b)^2}$. It is a
fraction and we see that its denominator is the same as the previous
expression. This is also true for the MathML representation:

\begin{lstlisting}[language=XML,caption={TODO},label={}]
<math xmlns="http://www.w3.org/1998/Math/MathML">
  <mfrac>
    <mn>3</mn>
    <msup>
      <mrow>
        <mo>(</mo>
        <mrow>
          <mi>a</mi>
          <mo>+</mo>
          <mi>b</mi>
        <mrow>
        <mo>)</mo>
      </mrow>
      <mn>2</mn>
    </msup>
  </mfrac>
</math>
\end{lstlisting}

\ \\


Token Elements

\ \\

Token elements are needed for representing tokens: the smallest units of
mathematical notation that convey some meaning.

There are several token elements:

\begin{itemize}
\itemsep1pt\parskip0pt\parsep0pt
\item   \texttt{mi} identifier
\item   \texttt{mn} number
\item   \texttt{mo} operator, fence, or separator
\item   \texttt{mtext} text
\item   \texttt{mspace} space
\item   \texttt{ms} string literal
\end{itemize}

Often tokens are just single characters, like
\texttt{<mi>E</mi>} or
\texttt{<mn>5</mn>}, but
there are cases when tokes are multi-character, e.g.
\texttt{<mi>sin</mi>} or
\texttt{<mi>span</mi>}.

In MathML \texttt{mi} elements represent some symbolic name or text that
should be rendered as identifiers. Identifiers could be variables,
function names, and symbolic constants.

Transitional mathematical notation often involve some special
typographical properties of fonts, e.g. using bold symbols e.g.
$\mathbf x$ to denote vectors or capital script symbols
e.g. $\mathcal G$ to denote groups and sets. To address
this, there is a special attribute ``mathvariant'' that can take values
such as ``bold'', ``script'' and others.

Numerical literals are represented with \texttt{mn} elements. Typically
they are sequences of digits, sometimes with a decimal point,
representing an unsigned integer or real number, e.g.
\texttt{<mn>50</mn>} or
\texttt{<mn>50.00</mn>}.

Finally, operators are represented with \texttt{mo} elements. Operators
are ...


\ \\

Layouts

Layout elements are needed to form complex mathematical expressions from
simple ones. They group elements in some particular way. For example:

\begin{itemize}
\itemsep1pt\parskip0pt\parsep0pt
\item
  \texttt{mrow} groups any number of sub-expressions horizontally
\item
  \texttt{mfrac} form sa fraction from two sub-expressions
\item
  \texttt{msqrt} forms a square root (radical without an index)
\end{itemize}

Some layout elements are used to add subscripts and superscripts:

\begin{itemize}
\itemsep1pt\parskip0pt\parsep0pt
\item
  \texttt{msub} attach a subscript to a base
\item
  \texttt{msup} attach a superscript to a base
\item
  \texttt{msubsup} attach a subscript-superscript pair to a base
\end{itemize}

And special kinds of scripts (TODO: describe in more details)

\begin{itemize}
\itemsep1pt\parskip0pt\parsep0pt
\item
  \texttt{munder} attach an underscript to a base
\item
  \texttt{mover} attach an overscript to a base
\item
  \texttt{munderover} attach an underscript-overscript pair to a base
\end{itemize}

For example, $\vec v$ will be rendered as

\begin{lstlisting}[language=XML,caption={TODO},label={}]
<math xmlns="http://www.w3.org/1998/Math/MathML">
  <mover>
    <mi>v</mi>
    <mo>&rarr;</mo>
  </mover>
</math>
\end{lstlisting}


This is how we would represent $\hat{ \mathbf x}$ (a bold x with a hat) in MathML:

\begin{lstlisting}[language=XML,caption={TODO},label={}]
<math xmlns="http://www.w3.org/1998/Math/MathML">
  <mrow>
    <mover>
      <mrow>
        <mi mathvariant="bold">x</mi>
      </mrow>
      <mo>&#x005E;<!-- ^ --></mo>
    </mover>
  </mrow>
</math>
\end{lstlisting}

There are more complex elements such as \texttt{mtable}.

MathML presentation elements only suggest specific ways of rendering

Math Entities

Certain characters are used to name identifiers or operators that in
traditional notation render the same as other symbols or usually
rendered invisibly.

entities \texttt{\&InvisibleTimes;} \texttt{\&rarr;}

The complete list of MathML entities is described in [Entities].




\subsection{Math-aware POS tagging} \label{sec:postagging}
Part-of-Speech Tagging (POS Tagging) is a typical Natural Language Processing 
task which assigns a POS Tag to each word in a given text \cite{jurafsky2000speech}. 
While the POS Tagging task is mainly a tool for text processing, it can 
also be applicable to scientific documents with mathematical expressions, 
and can be adjusted to dealing with formulae. 

A \emph{POS tag} is an abbreviation that corresponds to some
part of speech. Penn Treebank POS Scheme \cite{santorini1990part} is 
a commonly used POS tagging scheme which defines a set of part-of-speech tags 
for annotating English words.
For example, \texttt{JJ} is an adjective (``big''), \texttt{RB} as in adverb, 
\texttt{DT} is a determiner (``a'', ``the''), \texttt{NN} is a 
noun (``corpus'') and \texttt{SYM} is used for symbols (``$>$'', ``$=$''). 


However the Penn Treebank scheme does not have special tags for mathematics,
but it is flexible enough and can be extended to include additional tags.
For example, we can include a math-related tag \texttt{MATH}.
Usually it is done by first applying traditional POS taggers (like Stanford
CoreNLP \cite{manning2014stanford}), and then 
refining the results by re-tagging math-related tokens of text as \texttt{MATH}
\cite{schoneberg2014pos}.


For example, consider the following sentence:
``The relation between energy and mass is
described by  the mass-energy equivalence formula $E = mc^2$,
where $E$ is energy, $m$ is mass and $c$ is the speed of light''.
In this case we will assign the MATH tag to ``$E = mc^2$'', ``$E$'',
``$m$'' and ``$c$''

However we can note that for finding identifier-definition relations 
the \texttt{MATH} tag alone is not sufficient: we need to distinguish 
between complex mathematical expressions and stand-alone identifiers - 
mathematical expressions that contain only one symbol: the identifier.
For the example above we would like to be able to distinguish the 
expression ``$E = mc^2$'' from identifier tokens ``$E$'',
``$m$'' and ``$c$''. Thus we extend the Penn Treebank scheme even more 
and introduce an additional tag \texttt{ID} to denote stand-alone identifiers. 


Thus, in the example above ``$E = mc^2$'' will be assigned the \texttt{MATH} tag
and ``$E$'',``$m$'' and ``$c$'' will be annotated with \texttt{ID}.

In the next section we discuss how this can be used to find identifier-definition 
relations.


\subsection{Extraction Methods} \label{sec:definition-extraction-methods}
There are several ways of extracting the identifier-definition relations.
Here we will review the following:

\begin{itemize}
\itemsep1pt\parskip0pt\parsep0pt
  \item Nearest Noun
  \item Pattern Matching
  \item Machine-Learning based methods
  \item Probabilistic methods
\end{itemize}


\subsubsection{Nearest Noun Method}

\ \\

The Nearest Noun  \cite{grigore2009towards} \cite{yokoi2011contextual} 
is the simplest definition extraction method.
It assumes that the definition is a combination of ad
It finds definitions by looking for combinations of adjectives and nouns
(sometimes preceded by determiners) in the text before the identifier. 

I.e. if we see a token annotated with \texttt{ID}, and then a sequence 
consisting only of adjectives (\texttt{JJ}), nouns (\texttt{NN}, \texttt{NNS}) 
and determiners (\texttt{DET}), then we say that this sequence is 
the definition for the identifer. 

For example, given the sentence ``In other words, the bijection $\sigma$ normalizes
$G$ in ...'' we will extract relation ($\sigma$, "bijection").



\subsubsection{Pattern Matching Methods} \label{sec:pattern-matching}

\ \\

The Pattern Matching method \cite{quoc2010mining} is an extension of the 
Nearest Noun method: in Nearest Noun we are looking for one specific patten 
where identifier is followed by the definition, but we can define several such 
patterns and use them to extract definitions. 

For example, we can define the following patterns:

\begin{itemize}
\itemsep1pt\parskip0pt\parsep0pt
  \item \texttt{IDE} \texttt{DEF}
  \item \texttt{DEF} \texttt{IDE}
  \item let$|$set \texttt{IDE} denote$|$denotes$|$be \texttt{DEF}
  \item \texttt{DEF} is$|$are denoted$|$\texttt{def}ined$|$given as$|$by \texttt{IDE}
  \item \texttt{IDE} denotes$|$denote$|$stand$|$stands as$|$by \texttt{DEF}
  \item \texttt{IDE} is$|$are \texttt{DEF}
  \item \texttt{DEF} is$|$are \texttt{IDE}
  \item and many others
\end{itemize}


In this method \texttt{IDE} and \texttt{DEF} are placeholders that are 
assigned a value when the pattern is matched against some subsequence 
of tokens.  \texttt{IDE} and DEF need to 
satisfy certain criteria in order to be successfully matched: like in the 
Nearest Noun method we assume that \texttt{IDE} is some token annotated with 
\texttt{ID} and \texttt{DEF} is a phrase containing adjective (\texttt{JJ}), 
nouns (\texttt{NN}) and  determiners (\texttt{DET}). Note that the first patten corresponds 
to the Nearest Noun pattern.

The patterns above are combined from two sources: one 
is extracted from a guide to writing mathematical papers in English 
(\cite{trzeciak1995writing}) by \textbf{TODO}, and another 
is extracted from Graphs and Combinatorics papers from Springer 
by \textbf{TODO}. 


The pattern matching method is often used as the baseline method 
for identifier-definition extraction methods \cite{kristianto2012extracting}
\cite{kristianto2014extracting} \cite{pagael2014mlp}.


\subsubsection{Machine Learning Based Methods}

\ \\

The definition extraction problem can be formulated as a binary classification
problem: given a pair (identifier, candidate-definition), does this pair 
correspond to real identifier-definition relation? 

To do this we find all candidate pairs: identifiers are tokens 
annotated with \texttt{ID}, and candidate defections are nouns and 
noun phrases from the same sentence as the definition. 

Once the candidate pairs are found, we extract the following features \cite{kristianto2014extracting} \cite{yokoi2011contextual}:


\begin{itemize}
\itemsep1pt\parskip0pt\parsep0pt
  \item boolean features for each of the patterns from 
    section~\ref{sec:pattern-matching} indicating if the pattern is matched
  \item indicator if there's a colon or comma between candidate and identifier
  \item indicator if there's another math expression between candidate and identifier
  \item indicator if candidate is inside parentheses and identifier is outside
  \item distance (in words) between the identifier and the candidate
  \item the position of candidate relative to identifier
  \item text and POS tag of one/two/three preceding and following tokens around the candidate
  \item text of the first verb between candidate and identifier
  \item many others
\end{itemize}

Once the features are extracted, a binary classifier can be trained to predict
if an unseen candidate pair is a relation or not. 
For this task the popular choices of classifiers are Support Vector Machine 
classifier with linear kernel \cite{kristianto2014extracting} \cite{yokoi2011contextual}
and Conditional Random Fields \cite{kristianto2014extracting},
but, in principle, any other binary classifier can be applied 
as well.


\subsubsection{Probabilistic Approaches} \label{sec:mlp}

\ \\

In the Mathematical Language Processing approach \cite{pagael2014mlp}
an identifier-definition relation is seen as a pair $\langle$identifier, 
probability distribution over definition candidates$\rangle$. Thus, for
definition extraction the candidate definitions are ranked by
their probability of defining the identifier, 
and then only most probable candidates are retained.

The main idea of this approach is that the definitions occur very closely 
to identifiers in sentences, and the closeness can be used to
model the probability distribution over candidate definitions. 

There are two assumptions:

\begin{itemize}
\itemsep1pt\parskip0pt\parsep0pt
  \item identifier and its definition can only occur in the same sentence, 
      so the candidates are definitions are taken only from the sentence where 
      the identifier occurs (as in the Machine Learning approach)
  \item definitions are more likely to occur closer to the formula 
      where the identifier is used
\end{itemize}

With these assumptions in mind, the distribution over definition candidates 
can be modeled with:

\begin{itemize}
\itemsep1pt\parskip0pt\parsep0pt
  \item probability distribution $R_{\sigma_d}\big(\Delta(i, t) \big)$ where $\Delta(i, t)$
        is the distance between identifier $i$ and candidate definition $t$, 
        parameterized with $\sigma_d$,
  \item probability distribution $R_{\sigma_s}\big(n(i, t) \big)$ where $n(i, t)$ 
        is the number of sentences between the formula where $i$ is used 
        and the sentence where $t$ occurs, 
        parameterized with $\sigma_s$,
        and
  \item term frequency $\text{tf}(t, s)$: how many times $t$ occurs in the sentence $s$
\end{itemize}


All three elements can be combined together for ranking candidate definitions:
$$R(n, \Delta, t, d) = \cfrac{\alpha \, R_{\sigma_d}(\Delta) + \beta \, R_{\sigma_s}(n) + \gamma \, \text{tf}(t, s)}{\alpha + \beta + \gamma}$$
where $\alpha, \beta, \gamma$ are weighting parameters.

Instead of taking the raw distances, the Gaussian distribution is used
to model the fact that the probability of being the definition does not
decrease linearly as we move away from the identifier, but rather
exponentially. Thus, the distances are modeled with
$$R_{\sigma}(\Delta) = \exp \left( -\cfrac{1}{2} \cdot \cfrac{\Delta^2 - 1}{\sigma^2} \right)$$
assuming that the probability to find the relation at $\Delta = 1$ is maximal. 

There are several parameters in this model: $\sigma_d$ and $\sigma_s$
for controlling the widths of $R_{\sigma_d}(\Delta)$ and $R_{\sigma_s}(n)$
respectively; and  $\alpha, \beta, \gamma$  are parameters for controlling 
the contribution of different ranking components. 

The parameter $\sigma_d$ is the standard deviation of Gaussian that
models the distance to definition candidate. By examining several mathematical
articles manually it has been established that 
$R_{\sigma_d}(1) \approx 2 \cdot R_{\sigma_d}(5)$, i.e. it is two times more 
likely to find the real definition at distance  $\Delta=1$
rather than at distance $\Delta=5$, and thus $\sigma_d = \sqrt{ 12 / \ln 2}$.

The parameter $\sigma_s$ is the standard deviation of the Gaussian that models 
the distance (in the number of sentences) between the candidate definition 
and the formula where the identifier is used. Manual evaluation has shown that 
$\sigma_s = 2 \sqrt{ 1 / \ln 2}$. 

Finally, for weights $\alpha, \beta, \gamma$ the following parameters were chosen in 
\cite{pagael2014mlp}: $\alpha = \beta = 1$ and $\gamma = 0.1$.


\subsection{Performance Measures} \label{sec:definition-evaluation}

The common way to evaluate the performance of an Information Retrieval system
is to use Precision and Recall \cite{manning2008introduction}. 
\emph{Precision} is typically defined as is the fraction of retrieved documents 
that are relevant, while \emph{recall} is defined as 
the fraction of relevant documents that are retrieved. These performance measures 
can be adapted to measure the quality of Formula Extraction systems \cite{pagael2014mlp} 
as:

\begin{itemize}
\itemsep1pt\parskip0pt\parsep0pt
  \item \emph{Precision}: the fraction of definitions that are correct among extracted. 
        It is calculated as the number of correctly extracted definitions
        divided by the total number of extracted definitions. 
  \item \emph{Recall}: the fraction of definitions correctly extracted among all correct
        definitions. 
        It is calculated by dividing the number of correctly extracted definitions
        on the total number of identifiers with definition.
\end{itemize}


A common way of incorporating two measures into a single one is $F_\beta$-score
defined as $F_\beta = \frac{(\beta^2 +1)\, P \, R}{\beta^2 P + R}$ where 
$P$ and $R$ are precision and recall respectively, and $\beta$ is the trade-off
parameter. $F_1 = \frac{2PR}{P + R}$ is the balanced $F$ score when both 
precision and recall have equal weights.