\section{Namespace Discovery}

In this chapter we introduce the problem of namespace discovery in 
mathematical notation and suggest how this problem can be approached. 

First, we extend the idea of namespaces to mathematics in section~\ref{sec:math-namespaces},
and discuss the problem of namespace discovery in section~\ref{sec:discovery-namespaces},
and then argues that it is possible to use document cluster analysis 
to solve the problem in section~\ref{sec:clusters-namespaces}. 
Finally, we propose a way of representing identifiers in a vector space 
in section~\ref{sec:ism}.


\subsection{Namespaces in Mathematical Notation} \label{sec:math-namespaces}

The idea of namespaces can be extended to identifiers in mathematical
formulae. 

In mathematics and other sciences, formulae are used to communicate the results 
to other scientists. An \emph{identifier} is a symbol used in a mathematical 
formula and it typically has some semantic meaning. For example, in a formula 
$E = m c^2$ there are three identifiers: ``$E$'', ``$m$'' and ''$c$''.
\emph{Mathematical notation} is a system of naming identifiers in
mathematical formulae, and for each identifier in the formula the notation assigns
a precise semantic meaning \cite{wikinotation}. For example, in the expression
$E = mc^2$ the notation assigns
unambiguous meaning to the symbols ``$E$'', ``$m$'' and ``$c$'', and the
meaning of these symbols is recognized among physicists.

However, notations may conflict. For example, while it is common to use
symbol ``$E$'' to denote ``Energy'' in Physics, it also is used in Probability and
Statistics to denote ``Expected Value'', or in Linear Algebra to denote
``Elimination Matrix''.
We can compare the conflict of notations with the name collision problem
in namespaces, and try to address this problem by extending the notion of
namespaces to mathematical notation.

Thus, let us define a \emph{notation} $\mathcal N$ as a set of pairs $\{ (i, s) \}$,
where $i$ is a symbol or \emph{identifier} and $s$ is its semantic meaning
or \emph{definition}, such that for any pair $(i, s) \in \mathcal N$ there
does not exist another pair $(i', s') \in \mathcal N$ with $i = i'$.
Two notations $\mathcal N_1$ and $\mathcal N_2$
\emph{conflict}, if there exists a pair $(i_1, s_1) \in \mathcal N_1$ and a pair
$(i_2, s_2) \in \mathcal N_2$ such that $i_1 = i_2$ and $s_1 \ne s_2$.

Then we can define \emph{namespace} as a named notation. For example,
$\mathcal N_\text{physics}$ can refer to the notation used in Physics.
For convenience, in this work we can use the Java syntax to refer to
specific entries of a namespace. If $\mathcal N$ is a namespace and $i$
is an identifier such that $(i, s) \in \mathcal N$ for some $s$, then
``$\mathcal N$.$i$'' is a \emph{fully qualified name} of the identifier $i$ that
relates $i$ to the definition $s$. For example,
given a namespace $\mathcal N_\text{physics} =$ $\{ (E, \text{``energy''}),$
$(m, \text{``mass''}),$ $(c, \text{``speed of light''}) \} \big)$,
``$\mathcal N_\text{physics}$.$E$'' refers to ``energy'' -- the definition of ``$E$'' in the
namespace ``Physics''.

% TODO extend it to hierarchies ?

Analogously to namespaces in Computer Science, formally a mathematical namespace
can contain any set of identifier-definition pairs that satisfies the definition of
the namespace, but typically namespaces of mathematical notation
exhibit the same properties as well-designed software packages: they
have low coupling and high cohesion, meaning that all definitions
in a namespace
come from the same area of mathematical knowledge and the definitions
from different namespace do not intersect heavily.

However, mathematical notation does not yet exist in isolation, and it is 
usually observed indirectly by its usage in documents. To account for this fact,
we need to introduce a document-centric view on mathematical
namespaces: suppose we have a collection of $n$ documents
$\mathcal D = \{ d_1, d_2, \ ... \ , d_n \}$ and a set of $K$ namespaces
$\{\mathcal N_1, \mathcal N_2, \ ... \ ,$ $\mathcal N_K \}$.
A document $d_j$ can use a namespace $\mathcal N_k$ by \emph{importing}
identifiers from it. To import an identifier, the document uses an import statement
where the identifier $i$ is referred by its fully qualified name.
For example, a document ``Energy-mass equivalence'' would import
``$\mathcal N_\text{physics}$.$E$'', ``$\mathcal N_\text{physics}$.$m$'',
and ``$\mathcal N_\text{physics}$.$c$'', and then these identifiers can be used in
formulae of this document unambiguously.

A namespace exhibits low coupling if it is used only in a small
subset of documents, and high cohesion if all the documents in this subset
are related to the same domain.

But in real-life scientific document there are no import statements
in the document preamble, and they contain only natural language
texts along with some mathematical formulae. Yet we may still assume
that these import exists, but they are implicit, i.e. they are latent
and cannot be observed directly. Additionally, the namespaces themselves are
also not observed.

Typically in mathematical texts, when an identifier is first introduced,
its definition is given in the natural language description that surrounds
the formula. This description can be extracted and used to assign the meaning to
the identifiers. Once identifier definitions are extracted, a document
can be represented as a set of identifier-definition pairs,
and these pairs can be used to discover the namespaces.


In the next section we discuss how this problem can be addressed.


\subsection{Discovery of Identifier Namespaces} \label{sec:discovery-namespaces}

There are several ways of constructing a set of namespaces given a collection 
of documents. 

It is possible to do it manually by pre-defining a set of namespaces and 
then by manually assigning each identifier-definition relation to some of these
namespace. It is not only time consuming, but also very difficult: one has 
to know where to put the identifiers, and set of namespaces needs to be exhaustive. 

Alternatively, it can be done automatically, and in this work we suggest a different 
approach: use Machine Learning techniques for discovering namespaces automatically.

We illustrate our idea by first drawing an analogy between identifier
namespaces and namespaces in programming languages. In a well-designed application, 
we can distinguish between two types of application packages \cite{evans2004domain}:

\begin{itemize}
  \item \emph{type 1}: domain-specific packages that deal with one particular
    concept or domain area, and
  \item \emph{type 2}: packages that use other packages of
    the first type
\end{itemize}

% \textbf{TODO: you could use a real example here}


For example, for an application \verb|org.company.app|
there can be several domain-specific packages: \verb|org.company.app.domain.user|
with classes related to users, \verb|org.company.app.domain.account|
with classes related to user accounts, and a system-related package
\verb|org.company.app.tools.auth| that deals with authentication and
authorization. Then we also have a package \verb|org.company.app.web.manage|,
which belongs to the type 2: it handles web requests
while relying on classes from packages \verb|user| and \verb|account| to
implement the business logic and on \verb|auth| for making sure the
requests are authorized.

We can observe that the type 1 packages are mostly self-contained
and not highly coupled between each other, but type 2 packages mostly
use other packages of type 1: they depend on them.

This idea can be extended on the document-centric view on
identifier namespaces. Each document can be seen as a class that
imports identifiers defined in other documents.
Then the documents can be grouped together based on the identifiers
and the definitions they have, and then among these groups
there are some groups of documents that are of \emph{type 1}
and the rest are of \emph{type 2}. The type 1 document groups
contain information about closely related concepts, and they are
very homogenous (they have high cohesion), and they are also
not highly coupled with other document groups.
By using the import metaphor, we can say that the type 1 document groups
import only from few closely related namespaces.
Other documents are of \emph{type 2} and they do not have
low coupling: they are not very homogenous and they import from several namespaces

With this intuition we can refer to \emph{type 1} document groups
as \emph{namespace defining} groups. These groups can be seen as ``type 1''
packages: they define namespaces that are used by other \emph{type 2}
document groups. Once the namespace defining groups are found,
we can learn the namespace of these document.

Thus we need to find groups of homogenous documents given a
collection, and this is exactly what Cluster Analysis methods do.

In the next section we will argue why we can use traditional
document clustering techniques and what are the characteristics
that texts and identifiers have in common.


\subsection{Namespace Discovery by Cluster Analysis} \label{sec:clusters-namespaces}

We argue that cluster analysis techniques developed for text documents
should also work for cases when documents are represented by
identifers they contain.

The reason for this is that identifiers can be seen as ``words'' in 
the mathematical language, their senses are by described their definitions,
and the ``sentences'' of this language are formulae. 
Because identifiers are used like words, we can make the same assumptions 
about them. For example, words are distributed according to a power low distribution
\cite{manning2008introduction}, and therefore we can assume that identifiers 
also follow some power low. 

Additionally, natural languages suffer from lexical problems of variability
and ambiguity, and the two main problems are synonymy and polysemy
\cite{deerwester1990indexing} \cite{gliozzo2009semantic}:

\begin{itemize}
\itemsep1pt\parskip0pt\parsep0pt
  \item two words are \emph{synonymous} if they have the same meaning
        (for example, ``graph'' and ``chart'' are synonyms),
  \item a word is \emph{polysemous} is it can have multiple meanings
        (for example, ``trunk'' can refer to a part of elephant or a part of a car).
\end{itemize}

Note that identifiers have the same problems. For example,
``$E$'' can stand both for ``Energy'' and ``Expected value'',
so ``$E$'' is polysemous.

These problems have been studied in Information Retrieval and
Natural Language Processing literature.
One possible solution for the polysemy problem is \emph{Word Sense Disambiguation}
\cite{jurafsky2000speech}: either replace a word with its sense
\cite{stokoe2003word} or append the sense to the word. For example,
if the polysemous word is ``bank'' with meaning ``financial institution'',
then we replace it with ``bank\_finance''. The same idea can be used
for identifiers, for example if we have an identifier ``$E$'' which is
defined as ``energy'', then ``$E$'' can be replaced with ``$E$\_energy''.

Thus we see that text representation of documents and identifier representation
of documents have many similarities and therefore we can apply the set of
techniques developed for text representation for clustering documents based
on identifiers.

For document clustering, documents are usually represented using
Vector Space Models \cite{oikonomakou2005review} \cite{aggarwal2012survey}.
Likewise, we can introduce ``Identifier Vector Space Model'' analogously to
Vector Space Models for words, and then we can apply clustering algorithm
to documents represented in this space.


\subsection{Identifier Vector Space Model} \label{sec:ism}

The Vector Space Model discussed in section \ref{sec:vsm} can be adjusted to 
represent documents by identifers they contain instead of words. 
To do that we replace the vocabulary $\mathcal V$
with  a set of identifiers $\mathcal I = \{ i_1, i_2, \ ... \ , i_m \}$,
but documents are still represented as $m$-vectors $\mathbf d_j = (w_1, w_2, \ ... \ , w_m)$,
where $w_k$ is a weight of identifier $i_k$ in the document $\mathbf d_j$.
Likewise, we can define an identifier-document matrix $D$ as a matrix where
columns are document vectors and rows are indexed by the identifiers.

Identifiers, as terms, suffer from the problems of synonymy and polysemy,
and we solve this problem by extracting definitions for all the identifiers.
There are several ways of incorporating the extracted definitions into the
model:

\begin{itemize}
\itemsep1pt\parskip0pt\parsep0pt
  \item do not include definition information at all, use only identifiers;
  \item use ``weak'' identifier-definition association: include identifiers and
        definitions as separate dimensions;
  \item use ``strong'' association: append definition to identifier.
\end{itemize}

To illustrate how it is done, consider three relations ($E$, ``energy''),
($m$, ``mass'') and ($c$, ``speed of light''), and three documents
$d_1 = \{E, m, c\}, d_2 = \{ m, c\}, d_3 = \{ E \}$. Then

\begin{itemize}\itemsep1pt\parskip0pt\parsep0pt
  \item no definitions: dimensions are ($E$, $m$, $c$) and the identifier-document matrix is
  $$D = \left[
    \begin{array}{c|ccc}
       & d_1 & d_2 & d_3 \\
      \hline
      E & 1 & 0 & 1  \\
      m & 1 & 1 & 0 \\
      c & 1 & 1 & 0 \\
    \end{array}
  \right];$$
  \item ``weak'' association: dimensions are ($E$, $m$, $c$, energy, mass,
  speed of light), and the matrix is $$D = \left[
    \begin{array}{r|ccc}
       & d_1 & d_2 & d_3 \\
      \hline
      E                     & 1 & 0 & 1  \\
      m                     & 1 & 1 & 0 \\
      c                     & 1 & 1 & 0 \\
      \text{energy}         & 1 & 0 & 1  \\
      \text{mass}           & 1 & 1 & 0 \\
      \text{speed of light} & 1 & 1 & 0 \\
    \end{array}
  \right];$$
  \item ``strong'' association: dimensions are ($E$\_energy, $m$\_mass, $c$\_speed of light), and the matrix is $$D = \left[
    \begin{array}{r|ccc}
       & d_1 & d_2 & d_3 \\
      \hline
      E\text{\_energy} & 1 & 0 & 1  \\
      m\text{\_mass} & 1 & 1 & 0 \\
      c\text{\_speed of light} & 1 & 1 & 0 \\
    \end{array}
  \right].$$
\end{itemize}

Once a collection of documents is represented is some Identifier Vector Space, we 
can apply document clustering techniques discussed in the section~\ref{sec:doc-clustering}.
