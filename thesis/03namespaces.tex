\section{Namespaces as Document Clusters}


\subsection{Discovering Namespaces with Cluster Analysis}

Why it will work?

List properties of text that identifier-based representation of documents share as well

list characteristics of textual data



\begin{itemize}
  \item dimensionality is very large, but vectors are very sparse.
    e.g. vocabulary size $| V | = 10^5$, but documents may contain only 500 distinct words,
    or even less - when we consider sentences or tweets
  \item lexicon of document may be large, but words are typically correlated with each other
    so number of concepts ("principal components") is $\ll | V |$
  \item number of words across different documents may wary a lot
  \item word distributions follow Power Laws (Zipf's law etc - cite)
\end{itemize}

Problems in text:

\begin{itemize}
  \item problems of polysemy, homonymy and synonymy: semantic relations between words
  \item Polysemy is the capacity for a sign (such as a word, phrase, or symbol) to have multiple meanings (multiple senses)
  \item The state of being a homonym is called homonymy. In linguistics, a homonym is, in the strict sense, one of a group of words that share the same spelling and pronunciation but have different meanings.[1] http://dictionary.reference.com/browse/homonym
  \item Words that are synonyms are said to be synonymous, and the state of being a synonym is called synonymy.
  \item The analysis of synonymy, polysemy, and hyponymy and hypernymy is vital to taxonomy and ontology in the information-science senses of those terms.
  \item Homonyms are words that have the same pronunciation and spelling, but have different meanings. For example, rose (a type of flower) and rose (past tense of rise) are homonyms.
\end{itemize}

This is also true for identifiers and they have the same problems (illustrate that)
These problems are studied in IR and NLP literature, so we can apply them for identifiers


Identifier spaces (need to define what it is - analogous to documents vector space models) have the same characteristics so we can apply vector space techniques

in VSM for word sense disambiguation often word meaning is attached e.g. "bank\_finances", can do the same e.g. "E\_energy"

Then we discover namespaces in the identifier namespaces - it's done by clustering document-identifier matrix


\textbf{How to deal with these problems?} Term Extraction techniques:
these techniques create "artificial" terms that aren't really terms - they are generated, and not the ones that actually occurred in the text
The original terms don't have the optimal dimensionality for document content representation
because of the problems of polysemy, homonymy and synonymy
so we want to find better representation that doesn't suffer from these issues


\textbf{Assumptions:}


\begin{enumerate}
 \item documents are "mixtures" of namespaces: they take identifiers from several namespaces
 \item there are some documents are more "pure" than others: they either take identifiers exclusively from one namespace or from few very related namespaces
 \item there's a correlation between a category of a document and the namespaces
the document uses
\end{enumerate}




Under these assumptions: can approximate namespaces discovery by finding groups of "pure" documents. can evaluate purity by using category information and retain only pure ones

(or: that there is a strong correlation between identifiers in
a document and the namespace of the document, and this correlation can be exploited to
categorize documents and thus discover namespaces)



\subsection{Vector Space Model} \label{sec:vsm}

Vector Space model: term selection, weighting schemes; how to build identifier space and include definition information

\cite{sebastiani2002machine} - \emph{classification}



TF-IDF


tf is normalized by idf

idf - reduces weights of terms that occur more frequently

When applied to identifiers: some identifiers like $x$ or $y$ occur very frequently and don't have much discriminating power


to ensure that document matching is done with more discriminative words
apply sub-linear transformation to to avoid the dominating effect of words that occur very frequently

\subsection{Identifier Space Model} \label{sec:ism}


There are three ways of incorporating the definition information into the identifier space.

Given ($\lambda$, regularization) and ($w$, weight vector)

\begin{itemize}
  \item use only identifier information. dimensions are ($\lambda$, $w$)
  \item use "weak" identifier-definition association:  dimensions are ($\lambda$, $w$, regularization, weight vector)
  \item use "strong" association:  dimensions are ($\lambda$\_regularization, $w$\_weight vector)
\end{itemize}




\subsection{Similarity Measures and Distances}

Once the documents are represented in some vector space, we need to 
define how to compare these documents to each other. There are two 
ways of doing this: using a similarity function that tells how similar 
two objects are (the higher values - the more similar the objects),
or using a distance function, sometimes called ``dissimilarity function'',
which is the opposite of similarity (the higher the values, the less similar 
the objects) 

The most commonly used distance function in vector spaces is Euclidean 
distance which corresponds to the geometric distance between two points 
in the space. For example, if we have two points $\mathbf x$ and 
$\mathbf z$ then the Euclidean distance is the
length of the line that connects these two points. 
It is defined as
$\| \mathbf x - \mathbf z \| = \sqrt{ (\mathbf x - \mathbf z)^T (\mathbf x - \mathbf z) } = \sqrt{\sum_i (x_i - z_i)^2}$. This distance is also often called 
$L_2$ distance.

But Euclidean distance is not always meaningful for high dimensional data. 

$\| \mathbf x - \mathbf z \|^2 = (\mathbf x - \mathbf z)^T (\mathbf x - \mathbf z) = 
\mathbf x^T \mathbf x - 2 \mathbf x^T \mathbf z + \mathbf z^T \mathbf z = 
\| \mathbf x \|^2 - 2 \mathbf x^T \mathbf z + \| \mathbf z \|^2$

We see that it also takes into account the length of each individual vector, 
and therefore distance between related and not related documents can be 
the same. 

Consider the following example (from \cite{ertoz2003finding}):
There are 4 data points in 10-dimensional space indexed by 
identifiers $A_1, \ ... \ , A_{10}$


\begin{tabular}{|c|cccccccccc|}
  \hline
~ &  $A_1$ &   $A_2$ &   $A_3$ &  $A_4$ &  $A_5$ &  $A_6$ &  $A_7$ &  $A_8$ &  $A_9$ &  $A_{10}$ \\
  \hline
$\mathbf p_1$ &  3 &  0 &  0 &  0 &  0 &  0 &  0 &  0 &  0 &  0 \\
$\mathbf p_2$ &  0 &  0 &  0 &  0 &  0 &  0 &  0 &  0 &  0 &  4 \\
$\mathbf p_3$ &  3 &  2 &  4 &  0 &  1 &  2 &  3 &  1 &  2 &  0 \\
$\mathbf p_4$ &  0 &  2 &  4 &  0 &  1 &  2 &  3 &  1 &  2 &  4 \\
  \hline
\end{tabular}


distance between $\mathbf p_1$ and $\mathbf p_2$ is $\| \mathbf p_1 - \mathbf p_2\| = 5$
distance between $\mathbf p_3$ and $\mathbf p_4$ is $\| \mathbf p_3 - \mathbf p_4\| = 5$
so Euclidean distance between these two vectors is the same!
 but suppose these vectors correspond to documents and words ([[Vector Space Models]])
 $\mathbf p_3$ and $\mathbf p_4$ must be more similar to each other than $\mathbf p_1$ and $\mathbf p_2$: $\mathbf p_3$ and $\mathbf p_4$ have 7 words in common whereas $\mathbf p_1$ and $\mathbf p_2$ have only 2


When the data is sparse it's better to use different measure of distance/similarity
we need to ignore records where both vectors have 0
for example:
[[Dot Product]] and [[Cosine Similarity]]
[[Jaccard Coefficient]]



\textbf{Jaccard similarity, jaccard distance}

\textbf{Dot product}
Dot product, or inner product, can also be seen as a similarity measure: the 



Geometric Definition
Let $\vec v \cdot \vec w$ denote the ''dot product'' between vectors $\vec v$ and $\vec w$
definition: $\vec v \cdot \vec w = \| \vec v \| \cdot \| \vec w \| \cdot \cos \theta$ where $\theta$ is the angle between $\vec v$ and $\vec w$
$\| \vec v \|$ denotes the length of $\vec v$
 if two vectors are perpendicular, then $\cos \theta = 0$ and thus $\vec v \cdot \vec w = 0$
 if they co-directional, then $\theta = 0$ and $\vec v \cdot \vec w = \| \vec v \| \cdot \| \vec w \|$
 consequently, we have $\vec v \cdot \vec v = \| \vec v \|^2$


Why does this geometric definition make sense?

consider vectors $\vec u$, $\vec v$ and $\vec w$
let $\vec v + \vec u = \vec w$ or $\vec u = \vec w - \vec v$
(ADD PROPER IMAGE WITH COSINE THEOREM 
http://habrastorage.org/files/d9f/8b1/073/d9f8b10734864b92bdcf9cf5ac92a0dc.png
)

$\| \vec u \|^2 =  \| \vec w - \vec v \|^2 = (\vec w - \vec v) \cdot (\vec w - \vec v) = \| \vec w \|^2 + \| \vec v \|^2 - 2 \cdot \vec w \vec v$
by the Cosine Theorem we know that
$\| \vec u \|^2 =  \| \vec w - \vec v \|^2 = \| \vec w \|^2 + \| \vec v \|^2 - 2 \cdot \| \vec w \| \cdot \| \vec v \| \cdot \cos \theta$
so $\| \vec w \| \cdot \| \vec v \| \cdot \cos \theta = \cfrac{1}{2} (\| \vec w \|^2 + \| \vec v \|^2 - \| \vec w - \vec v \|^2) = \cfrac{1}{2} (\| \vec w \|^2 + \| \vec v \|^2 - \| \vec w \|^2 - \| \vec v \|^2 + 2 \cdot \vec w \vec v) = \vec w \cdot \vec v$
thus $\vec w \cdot \vec v = \| \vec w \| \cdot \| \vec v \| \cdot \cos \theta$
i.e. the definition makes sense from the The Cosine Theorem point of view


For two vectors $\mathbf v, \mathbf w \in \mathbb R^n$ we define the dot product as $\mathbf v^T \mathbf w = \sum\limits_{i = 1}^n v_i w_i$

These definitions are equivalent


\textbf{Cosine similarity} is a [[Similarity Function]] that is often used in [[Information Retrieval]]
 it measures the angle between two vectors,  and in case of IR - the angle between two documents

recall the definition of the [[Dot Product]]: $\mathbf v \cdot \mathbf w = \| \mathbf v \| \cdot \| \mathbf w \| \cdot \cos \theta$
or, by rearranging get $\cos \theta = \cfrac{\mathbf v \cdot \mathbf w}{\| \mathbf v \| \cdot \| \mathbf w \|}$
so, let's define the cosine similarity function as $\text{cosine}(\mathbf d_1, \mathbf d_2) = \cfrac{\mathbf d_1^T \mathbf d_2}{\| \mathbf d_1 \| \cdot \| \mathbf d_2 \|}$
cosine is usually $[-1, 1]$, but document vectors (see [[Vector Space Model]]) are usually non-negative, so the angle between two documents can never be greater than 90 degrees, and for document vectors $\text{cosine}(\mathbf d_1, \mathbf d_2) \in [0, 1]$
min cosine is 0 (max angle: the documents are orthogonal)
max cosine is 1 (min angle: the documents are the same)


If documents have unit length, then cosine similarity is the same as [[Dot Product]]
$\text{cosine}(\mathbf d_1, \mathbf d_2) = \cfrac{\mathbf d_1^T \mathbf d_2}{\| \mathbf d_1 \| \cdot \| \mathbf d_2 \|} = \mathbf d_1^T \mathbf d_2$
thus we can "unit-normalize" document vectors $\mathbf d' = \cfrac{\mathbf d}{\| \mathbf d \|}$ and then compute dot product on them and get cosine
 this "unit-length normalization" is often called "cosine normalization" in IR



\textbf{ Cosine Distance}
for documents $\text{cosine}(\mathbf d_1, \mathbf d_2) \in [0, 1]$
it is max when two documents are the same
how to define a distance? distance function should become larger as elements become less similar
since maximal value of cosine is 1, we can define '''cosine distance''' as
$d_c(\mathbf d_1, \mathbf d_2) = 1 - \text{cosine}(\mathbf d_1, \mathbf d_2) = 1 -  \cfrac{\mathbf d_1^T \mathbf d_2}{\| \mathbf d_1 \| \cdot \| \mathbf d_2 \|}$


Let's check if cosine distance is a proper metric, i.e. it satisfies all the requirements
Let $D$ be the document space and $\mathbf d_1, \mathbf d_2 \in D$
 $d_c(\mathbf d_1, \mathbf d_2) \geqslant 0$: checks - 0 is minimum
$d_c(\mathbf d_1, \mathbf d_1) = 0$ checks - $1 - \cos 0 = 0$
 $d_c(\mathbf d_1, \mathbf d_2) = d_c(\mathbf d_2, \mathbf d_1)$: checks - angle is the same


What about the triangle inequality?
under certain conditions is doesn't hold (Korenius2007) - so it's not a proper metric



\textbf{Cosine and Euclidean Distance}
Euclidean distance $\| \mathbf d_1 - \mathbf d_2 \| = \sqrt{(\mathbf d_1 - \mathbf d_2)^T (\mathbf d_1 - \mathbf d_2)}$


There's a connection between Cosine Distance end Euclidean Distance
consider two unit-normalized vectors $\mathbf x_1 = \mathbf d_1 / \| \mathbf d_1 \|$ and $\mathbf x_2 = \mathbf d_1 / \| \mathbf d_1 \|$
$\| \mathbf x_1 - \mathbf x_2 \|^2 = (\mathbf x_1 - \mathbf x_2)^T (\mathbf x_1 - \mathbf x_2) = \mathbf x_1^T \mathbf x_1 - 2 \, \mathbf x_1^T \mathbf x_2 + \mathbf x_2^T \mathbf x_2 = \| \mathbf x_1 \|^2 - 2 \, \mathbf x_1^T \mathbf x_2 + \| \mathbf x_2 \|^2 = 2 - 2 \, \mathbf x_1^T \mathbf x_2$
if vectors are unit-normalized, cosine = dot product, so we have
$\| \mathbf x_1 - \mathbf x_2 \|^2 = 2 \, (1 -  \mathbf x_1^T \mathbf x_2) = 2 \, \big(1 - \text{cosine}(x_1, x_2)\big) = 2 \, d_c(x_1, x_2)$


It can also make some sense visually:


\url{https://habrastorage.org/files/f73/289/979/f732899792f246358649e89765cd88da.png}
recall the [[Cosine Theorem]]: $a^2 = b^2 + c^2 - 2 bc \cos \theta$
$b = c = 1$, so we have $a^2 = 2 \, (1 - \cos \theta)$


Thus we can use Euclidean distance and interpret it as Cosine distance


Korenius, Tuomo, Jorma Laurikkala, and Martti Juhola. "On principal component analysis, cosine and Euclidean measures in information retrieval." 2007.



\subsection{Inverted Index} \label{sec:index}

Inverted Index is a IR technique for being able to retrieve results faster. 
It also can be used for making clustering faster 

if $D$ is a document-term matrix, then the inverted index is build 
by considering $D^T$ and 

In [[Databases]], [[Indexing (databases)|indexing]] is needed to speed up queries
want to avoid full table scan
same is true for [[Information Retrieval]] and other [[Text Mining]]/[[NLP]] tasks
Inverted index is a way of achieving this, and it can be generalized to other forms of input, not just text


For IR
index is a partial representation of a document that contains the most important information about the document
usually want to find terms to index automatically




Inverted Index for Similarity Search 

Idea:
usually a document contains only a small portion of terms
so document vectors are very sparse
typical distance is cosine similarity - it ignores zeros. for cosine to be non-zero, two docs need to share at least one term
$D^T$ is the inverted index of the term-document matrix $D$


This, to find docs similar to $d$:
for each $w_i \in d$
let $D_i = \text{index}[w_i] - d$ be a set of documents that contain $w_i$ (except for $d$ itself)
then take the union of all $D_i$
calculate similarity only with documents from this union


Can be used in [[Document Clustering]] to speed up similarity computation



Posting List 
Build a dictionary: a "posting" list
for each word we store ids of documents that have this word
document are sorted by ids
http://slidewiki.org/upload/media/images/29/509.png?filter=Resize-width-550
source of picture: http://slidewiki.org/print/deck/339
sorting - because it's easier to take union: just merge the posting list




