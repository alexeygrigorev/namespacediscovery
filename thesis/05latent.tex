\section{Discovering Latent Semantics}


\textbf{How to deal with these problems?} Term Extraction techniques:
these techniques create "artificial" terms that aren't really terms - they are generated, and not the ones that actually occurred in the text
The original terms don't have the optimal dimensionality for document content representation
because of the problems of polysemy, homonymy and synonymy
so we want to find better representation that doesn't suffer from these issues


\subsection{Latent Semantic Analysis} \label{sec:lsa}

LSI paper \cite{deerwester1990indexing}

Using SVD for clustering IR results \cite{osinski2004lingo}


 Decomposition of Term-Document Matrix
We can decompose the matrix $D$
using [[SVD]] and we'll get [[Latent Semantic Analysis]]
LSA can also be done with [[Non-Negative Matrix Factorization]]



Latent Semantic Analysis (LSA) is an NLP method:

\begin{itemize}
\itemsep1pt\parskip0pt\parsep0pt
\item
  mathematical/statistical method for modeling the meaning of
  words/passages by analysis of text via extracting and inferring
  relations of expected contextual usage of words in texts
\item
  idea: words that are used in the same contexts tend to have the same
  meaning
\item
  it extracts and represents ``usage-in-context'' meaning of words and
  it gives a characterization of words meaning
\end{itemize}

LSA


brings up the latent structure of the vocabulary

\begin{itemize}
\itemsep1pt\parskip0pt\parsep0pt
\item
  LSA closely approximates how humans learn and understand meaning of
  words and similarity between words
\item
  it applies \href{Factor Analysis}{Factor Analysis} to texts to extract
  concepts and then clusters documents into similar categories based on
  factor scores
\item
  produces measures of word-word, word-passage, passage-passage
  relations via dimensionality reduction technique \url{SVD}
\item
  Similarity estimates derived by LSA are not just frequencies or
  co-occurrences counts: it can infer deeper relations: hence ``Latent''
  and ``Semantic''
\end{itemize}

\textbf{Limitations}

\begin{itemize}
\itemsep1pt\parskip0pt\parsep0pt
\item
  makes no use of words order, punctuation
\end{itemize}


\textbf{LSA Steps}

3 major steps in LSA \cite{evangelopoulos2012latent}

\begin{itemize}
\itemsep1pt\parskip0pt\parsep0pt
\item
  Prepare documents
\item
  Construct \href{Vector Space Models}{Term-Document matrix} $D$
\item
  Reduce dimensionality of $D$ via \url{SVD}
\end{itemize}


Representation: Term-Document Matrix

Construct a Term-Document Matrix $D$ using the
\href{Vector Space Model}{Vector Space Model}

\begin{itemize}
\itemsep1pt\parskip0pt\parsep0pt
\item
  typically rows of $D$ - terms, columns of $D$ -
  documents/passages,

  \begin{itemize}
  \itemsep1pt\parskip0pt\parsep0pt
  \item
    or sometimes, rows of $D$ are documents, columns of $D$ - terms
  \end{itemize}
\item
  not necessarily documents, it can have passages: paragraphs or entire
  texts
\item
  each cell - typically a frequency with which a word occurs in a doc
\item
  also apply weighting: TF or TF-IDF
\end{itemize}

SVD and Dimensionality Reduction

Let $D$ be an $t \times p$ Term-Passage matrix

\begin{itemize}
\itemsep1pt\parskip0pt\parsep0pt
\item
  $t$ rows are terms, $p$ columns are passages,
  $\text{rank } D = r$
\item
  then SVD decomposition is $D = T \cdot
  \Sigma \cdot P^T$
\item
  $T$ is $t \times r$
  \href{Orthogonal Matrix}{Orthogonal Matrix}, contains left singular
  vectors, corresponds to term vectors
\item
  $\Sigma$ is $r \times r$ a diagonal
  matrix of singular values
\item
  $P$ is $r \times p$
  \href{Orthogonal Matrix}{Orthogonal Matrix}, contains right singular
  vectors, corresponds to passage vectors
\item
  and then $T \sqrt\Sigma$ are loadings
  for terms and $P \sqrt\Sigma$ - for
  passages
\end{itemize}

Now reduce the dimensionality:

\begin{itemize}
\itemsep1pt\parskip0pt\parsep0pt
\item
  want to combine the surface text information into some deeper
  abstraction
\item
  finding the optimal dimensionality for final representation in the
  Semantic Space is important to properly capture mutual usage of words
\item
  the ``True Semantic Space'' should address the problems of ambiguity
\end{itemize}

So, Apply reduced-rank \url{SVD}

\begin{itemize}
\itemsep1pt\parskip0pt\parsep0pt
\item
  $D \approx T_k \cdot
  \Sigma_k \cdot P^T_k$
\item
  keep only $k$ largest singular values
\item
  the result: best $k$-dim approximation of the original matrix $D$
\item
  for NLP $k = 300 \pm 50$ usually works the best
\item
  but it should be \href{Parameter Tuning}{tuned} because it heavily
  depends on the domain
\end{itemize}

Semantic Space

LSA constructs a semantic space via SVD:

\begin{itemize}
\itemsep1pt\parskip0pt\parsep0pt
\item
  $T$ is $t \times r$
  \href{Orthogonal Matrix}{Orthogonal Matrix}, contains left singular
  vectors, corresponds to term vectors
\item
  $\Sigma$ is $r \times r$ a diagonal
  matrix of singular values
\item
  $P$ is $r \times p$
  \href{Orthogonal Matrix}{Orthogonal Matrix}, contains right singular
  vectors, corresponds to passage vectors
\item
  and then $T \sqrt\Sigma$ are loadings
  for terms and $P \sqrt\Sigma$ - for
  passages
\end{itemize}

Language-theoretic interpretation:

\begin{itemize}
\itemsep1pt\parskip0pt\parsep0pt
\item
  LSA vectors approximate:
\item
  the meaning of a word as its average effect of the meaning of passages
  in which they occur
\item
  the meaning of a passage as meaning of its words
\end{itemize}

After doing the SVD, we get the reduced space - this is the semantic
space

Examples

Let's consider titles of some articles example from \cite{landauer1998introduction} \cite{deerwester1990indexing}


TODO: change it to identifiers!

\begin{itemize}
\itemsep1pt\parskip0pt\parsep0pt
\item
  $c_1$: ``Human machine interface for ABC computer applications''
\item
  $c_2$: ``A survey of user opinion of computer system response
  time''
\item
  $c_3$: ``The EPS user interface management system''
\item
  $c_4$: ``System and human system engineering testing of EPS''
\item
  $c_5$: ``Relation of user perceived response time to error
  measurement''
\item
  $m_1$: ``The generation of random, binary, ordered trees''
\item
  $m_2$: ``The intersection graph of paths in trees''
\item
  $m_3$: ``Graph minors IV: Widths of trees and well-quasi-ordering''
\item
  $m_4$: ``Graph minors: A survey''
\end{itemize}

Matrix:

$$D =
\left[\begin{array}{c|ccccccccc}
 & c_1 & c_2 & c_3 & c_4 & c_5 & m_1 & m_2 & m_3 & m_4 \\
\hline
\text{human} & 1 & 0 & 0 & 1 & 0 & 0 & 0 & 0 & 0 \\
\text{interface} & 1 & 0 & 1 & 0 & 0 & 0 & 0 & 0 & 0 \\
\text{computer} & 1 & 1 & 0 & 0 & 0 & 0 & 0 & 0 & 0 \\
\text{user} & 0 & 1 & 1 & 0 & 1 & 0 & 0 & 0 & 0 \\
\text{system} & 0 & 1 & 1 & 2 & 0 & 0 & 0 & 0 & 0 \\
\text{response} & 0 & 1 & 0 & 0 & 1 & 0 & 0 & 0 & 0 \\
\text{time} & 0 & 1 & 0 & 0 & 1 & 0 & 0 & 0 & 0 \\
\text{EPS} & 0 & 0 & 1 & 1 & 0 & 0 & 0 & 0 & 0 \\
\text{survey} & 0 & 1 & 0 & 0 & 0 & 0 & 0 & 0 & 1 \\
\text{trees} & 0 & 0 & 0 & 0 & 0 & 1 & 1 & 1 & 0 \\
\text{graph} & 0 & 0 & 0 & 0 & 0 & 0 & 1 & 1 & 1 \\
\text{minors} & 0 & 0 & 0 & 0 & 0 & 0 & 0 & 1 & 1 \\
\end{array} \right] $$

Note:

\begin{itemize}
\itemsep1pt\parskip0pt\parsep0pt
\item
  row vectors for ``human'' and ``user'' are orthogonal: their dot
  product is zero, but they are supposed to be similar, so it must be
  positive
\item
  also, ``human'' and ``minors'' are orthogonal, but they are not
  similar, so it must be negative
\end{itemize}

Let's apply SVD:

\begin{itemize}
\itemsep1pt\parskip0pt\parsep0pt
\item
  $D = W \Sigma P$
\item
  2-dim approximation: $D_2 = W_2 \Sigma_2 P_2$
\end{itemize}

$$D_2 =
\left[\begin{array}{c|ccccccccc}
 & c_1 & c_2 & c_3 & c_4 & c_5 & m_1 & m_2 & m_3 & m_4 \\
\hline
\text{human} & 0.16 & 0.4 & 0.38 & 0.47 & 0.18 & -0.05 & -0.12 & -0.16 & -0.09 \\
\text{interface} & 0.14 & 0.37 & 0.33 & 0.4 & 0.16 & -0.03 & -0.07 & -0.1 & -0.04 \\
\text{computer} & 0.15 & 0.51 & 0.36 & 0.41 & 0.24 & 0.02 & 0.06 & 0.09 & 0.12 \\
\text{user} & 0.26 & 0.84 & 0.61 & 0.7 & 0.39 & 0.03 & 0.08 & 0.12 & 0.19 \\
\text{system} & 0.45 & 1.23 & 1.05 & 1.27 & 0.56 & -0.07 & -0.15 & -0.21 & -0.05 \\
\text{response} & 0.16 & 0.58 & 0.38 & 0.42 & 0.28 & 0.06 & 0.13 & 0.19 & 0.22 \\
\text{time} & 0.16 & 0.58 & 0.38 & 0.42 & 0.28 & 0.06 & 0.13 & 0.19 & 0.22 \\
\text{EPS} & 0.22 & 0.55 & 0.51 & 0.63 & 0.24 & -0.07 & -0.14 & -0.2 & -0.11 \\
\text{survey} & 0.1 & 0.53 & 0.23 & 0.21 & 0.27 & 0.14 & 0.31 & 0.44 & 0.42 \\
\text{trees} &-0.06 & 0.23 & -0.14 & -0.27 & 0.14 & 0.24 & 0.55 & 0.77 & 0.66 \\
\text{graph} &-0.06 & 0.34 & -0.15 & -0.3 & 0.2 & 0.31 & 0.69 & 0.98 & 0.85 \\
\text{minors} &-0.04 & 0.25 & -0.1 & -0.21 & 0.15 & 0.22 & 0.5 & 0.71 & 0.62 \\
\end{array}\right]$$


What's the effect of dimensionality reduction here?

\begin{itemize}
\itemsep1pt\parskip0pt\parsep0pt
\item
  words appear less or more frequent than originally
\item
  consider two cells: (``survey'', $m_4$) and (``trees'', $m_4$)
\item
  original document: 1 and 0
\item
  reduced document: 0.42 and 0.66
\item
  because $m_4$ contains ``graph'' and ``minor'', the 0 for ``trees''
  was replaced by 0.42 - they are related terms
\item
  so it can be seen as estimate of how many times word ``trees'' would
  occur in other samples that contain ``graph'' and ``minor''
\item
  the count for ``survey'' went down - it's not expected in this context
\end{itemize}

So in the reconstructed space:

\begin{itemize}
\itemsep1pt\parskip0pt\parsep0pt
\item
  dot product between ``user'' and ``human'' is positive
\item
  dot product between ``human'' and ``minors'' is negative
\item
  it tells us way better whether terms are similar or not even when they
  never co-occur together
\end{itemize}

Taking 2 principal components is the same as taking only 2 abstract
concepts

\begin{itemize}
\itemsep1pt\parskip0pt\parsep0pt
\item
  each word in the vocabulary has some amount of these 2 concepts (we
  see how much by looking at 1st and 2nd column of $W$)
\end{itemize}

The idea:

\begin{itemize}
\itemsep1pt\parskip0pt\parsep0pt
\item
  we don't want to reconstruct the underlying data perfectly, but
  instead we hope to find the correlation and the abstract concepts
\end{itemize}

Latent Semantic Analysis (LSA) $\approx$ Latent Semantic Indexing (LSI)

\begin{itemize}
\itemsep1pt\parskip0pt\parsep0pt
\item
  LSI is the alias of LSA for \href{Information Retrieval}{Information   Retrieval}
\item
  indexing and retrieval method that uses
  \href{Singular Value Decomposition}{SVD} to identify patterns in
  relations between terms and concepts
\item
  instead of literal match between query and documents (e.g. using
  cosine in the traditional vector space morels), convert both into the
  Semantic Space and calculate the cosine there
\end{itemize}



After doing dimenstionality reduction with LSA/SVD apply usual k-means (cite papers where
this approach is used)



\subsection{Semantic Domains}

The theory that connects linguistics and LSA \cite{gliozzo2009semantic}

Formal definition of Semantic Domain, why useful (why Semantic Domain may
be better concept for namespaces than Clusters),
Domain Spaces (sort of Semantic Spaces), how can be discovered with LSA




\subsection{Non-Negative Matrix Factorization} \label{sec:nmf}

There are many different factorization techniques for document clustering \cite{osinski2006improving}

One of them is NMF \cite{lee1999nnmf}

NMF can be directly interpreted as cluster assignment and has been
used for document clustering as an alternative to LSA \cite{xu2003document}





