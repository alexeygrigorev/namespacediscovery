\section{Implementation}

\subsection{Data set}
The data set of our choice is a dump of English wikipedia \cite{enwikidump}

For category information use DBPedia \cite{bizer2009dbpedia}

\subsection{Definition Extraction}


For extracting identifier-definition pair the MLP Project approach \cite{pagael2014mlp}
(see section~\ref{sec:mlp}).

Implemented in Apache Flink \cite{flink}

First, the original dataset is preprocessed enriched with enhanced MathML,
and then the dataset is filtered such that only articles with
the math tag are left.

Each document is in wiki XML. For each document we then extract the title,
and the content, and extract all the formulas from the content.
Some formulas can be typed directly with unicode - they will
be missed. The math tags are replaced with ``FORMULA\_hash'' placeholders and
and the content of the tags is kept separately from the document content.

The content of the formulas is processed: we are not interested in the
semantics of a formula, only in its identifiers. We extract the identifiers
by looking for all \verb|<ci>| tags. There are two types of identifiers:
simple identifiers (like $t$, $C$, $\mu$, etc) and identifiers with subscripts
(e.g. $x_1$, $\xi_i$ or even $\beta_{\text{slope}}$.
Superscripts are usually powers (e.g. $x^2$) and therefore they are not interesting
for us. There are exceptions to this (e.g. $\sigma^2$ is often an identifier on it
own) but these cases are rare and can be ignored.
The identifiers are extracted with XPath queries.

\begin{itemize}
  \item \verb|//m:mi[not(ancestor::m:msub)]/text()| for all \verb|<ci>| tags
that are not subscript identifers
  \item \verb|//m:msub| for subscript identifiers
\end{itemize}

Since we are only interested in the identifiers of each formula, we
keep them and discard the rest. As the result, we have a ``Bag of Formulas''.

Next all wiki markup (e.g. headers, tables, images etc). Only inner-wiki
links are kept: they can be useful as candidate definitions. This is done
using a Java library ``Mylyn Wikitext'' \cite{mylynwikitext}.


The next step is the NLP step. For NLP related tasks we use StanfordNLP
\cite{manning2014stanford}. First, we tokenize the text and split it
by sentences, and then we apply Math-aware POS tagging
(see section~\ref{sec:postagging}, page~\pageref{sec:postagging}).
For formulas and identifiers we introduce two new POS Classes: ``ID'' and ``MATH''.
The library does not know about these classes, so its output
goes through additional post-processing: if a token starts with ``FORMULA\_'',
then it is a placeholder for a math formula, and therefore we set its tag
to ``MATH''. If such a formula contains only one identifier, then the token
is replaced with this identifier, and we tag it as ``ID''.
Additionally, we keep a list of all identifiers found in the document,
and then for each token we check if this token is in this list.
If it is, then it is tagged with ``ID'' as well.

We also take advantage of wikipedia inner links: the text inside a link
can potentially be a candidate definition. To use them we introduce
another wiki-related POS Tag: ``LINK''. To find wiki links, we first find
all token subsequences that start with \verb|[[| and end with \verb|]]|.
Then these subsequences are concatenated and classified as ``LINK''.

Finally, we are interested in all sequence of successive nouns
(both singular and plural) possibly preceded by an adjective.
We concatenate all such subsequences into one token tagged with ``NOUN\_PHRASE''.

Thus, all tokens tagged with ``LINK'' and ``NOUN\_PRASE'' are definition candidates.
The next step is to select the most probable identifier-definition pairs, this is
done with the MLP approach (see section~\ref{sec:mlp} at page~\pageref{sec:mlp}).
The output of this step is a list of identifier-definition pairs. 


Most common definitions:

[(u'time', 2464), (u'length', 1325), (u'distance', 1246), (u'matrix', 1230), (u'form', 1199), (u'radius', 1115), (u'mass', 1080), (u'angle', 1030), (u'order', 951), (u'sequence', 913), (u'degree', 904), (u'field', 880), (u'space', 879), (u'group', 786), (u'probability', 777), (u'temperature', 771), (u'velocity', 753), (u'constants', 735), (u'dimension', 714), (u'density', 700)]

pairs:


time: $t$ (1086)
mass: $m$ (424)
angle: $\theta$ (421)
temperature: $T$ (400)
radius: $r$ (395)
velocity: $v$ (292)
density: $\rho$ (290)
group: $G$ (287)
volume: $V$ (284)
wavelength: $\lambda$ (263)
radius: $R$ (257)
degree: $n$ (233)
distance: $r$ (220)
speed of light: $c$ (219)
length: $L$ (216)
length: $n$ (189)
order: $n$ (188)
dimension: $n$ (185)
size: $n$ (178)
mass: $M$ (171)


\subsection{Data Cleaning}


Additional identifier
information like hats ($\hat w$, tildes $\tilde y$, bars $\bar X$ or arrows $\overrightarrow x$
are discarded: only the identifier name is kept.
Often different fonts are used to convey some semantic information about identifiers.
For example $\mathbf w$ is a vector, $\mathcal H$ is a set, $\mathbb R$ is a space.
Unfortunately there is no single notation for this and there is a lot of variance:
e.g. $\overrightarrow x$ vs $\boldsymbol x$ vs $\mathbf x$, $\mathbb R$ vs $\mathbf R$ vs $\mathfrak R$.
Therefore all this information is discarded, so $\mathbf w$ becomes just $w$
and $\mathfrak R$ becomes $R$.
In the wikipedia dataset this information is encoded directly into MathML
i.e. the symbols use special unicode characters to convey font features such as 
boldness, Fraktur, etc,
so it had to be normalized on the character level - by converting characters
from special unicode blocks (``Mathematical Alphanumeric Symbols'' block)
back to standard ASCII positions (``Basic Latin'' block).

A lot of noise (``false positives'')
gets captured inside \verb|<ci>| tags: many mathematic-related symbols
like ``\textasciicircum'', ``\#'',``$\forall$'', ``$\int$''; 
miscellaneous symbols like ``$\diamond$'' or
``$\circ$'' arrows like ``$\to$'' and ``$\Rightarrow$'', and special characters like
``$\lceil$''.

To filter out these one-symbol false positives we exclude the following unicode blocks:
``Spacing Modifier Letters'', ``Miscellaneous Symbols'', ``Geometric Shapes'', ``Arrows'',
``Miscellaneous Technical'', ``Box Drawing'', ``Mathematical Operators''
(except ``$\nabla$'' which is sometimes used as an identifier)
and ``Supplemental Mathematical Operators''.
Some symbols (like ``='', ``+'', ``\~'', ``\%'', ``?'', ``!'')
belong to commonly used unicode blocks 
which we cannot exclude altogether. For these symbols we manually 
prepared a stop list for filtering them.


It also captures multiple-symbol false positives: operators and functions
like ``sin'', ``cos'', ``exp'', ``max'', ``trace''; 
words commonly used in formulas like ``const'', ``true'', ``false'',
``vs'', ``iff''; English stop words like ``where'', ``else'', ``on'', 
``of'', ``as'', ``is''; units like ``mol'', ``dB'', ``mm''.
These false identifiers are excluded by a stop list as well: if a
candidate identifier is in the list, it is filtered out.


Additionally, many shortlisted definitions are either not valid definitions or too general. 
For example, some identifiers become associated with ``if and only if'', 
``alpha'', ``beta'', ``gamma'', which are not valid definitions. Other definitions
like ``vector'', ``element'', ``number'' or ``variable'' are valid, but 
they are too general and not descriptive.
We maintain a stop list of such false definitions and filter them out from 
the result. 




\subsection{Document Clustering}

First step is discarding all definitions (from each id-pair definitions)
that occur only once and therefore do not have much discriminating power. 

Then we need to build an Identifier Space (see section~\ref{sec:ism}). 
There are three ways of doing it.

\begin{itemize}
  \item keeping identifiers alone
  \item ``weak identifier-definition association'': putting identifiers along with definitions
  \item identifier-definition pairs 
\end{itemize}

In the first case we use only identifier information - and discard the 
definitions altogether.

In the second and third cases we keep the definitions. 
But there is some variability in the definitions. For example, 
the same identifier can be assigned to ``Cauchy stress tensor'' and
``stress tensor'' - which are almost the same thing. To reduce 
this variability we tokenize the definition and use individual tokens
as dimensions of the space. For example, if we have two pairs
($\sigma$, ``Cauchy stress tensor'') and ($\sigma$, ``stress tensor''),
for second case we have dimensions $(\sigma, \text{Cauchy}, \text{stress}, \text{tensor})$,
while for the third case we have 
$(\sigma\text{\_Cauchy}, \sigma\text{\_stress}, \sigma\text{\_tensor})$.


Additionally, to decrease the effect of variability further, 
a stemming technique is applied to each definition token. 
We use Snowball stemmer for English \cite{porter2001snowball} 
implemented in NLTK \cite{bird2006nltk}: 
a python library for Natural Language Processing. 

Each document is vectorized (converted to a vector form) by using
\verb|TfidfVectorizer| from  scikit-learn \cite{scikit-learn}. 
We use the following settings: 

- \verb|use_idf=True, min_df=2|
- \verb|use_idf=False, min_df=2|
- \verb|use_idf=False, sublinear_tf=True, min_df=2|

The output is a document-identifier matrix (analogous to ``document-term''):
documents are rows and identifiers/definitions are columns. 

The output of \verb|TfidfVectorizer| is row-normalized, i.e. 
all rows has unit length. 


K-Means and Mini-Batch K-Means implementation from scikit-learn \cite{scikit-learn}.
Classes \verb|KMeans|, \verb|MiniBatchKMeans|

If rows are unit-normalized, then running k-means with Euclidean distance
is equivalent to cosine distance (see section~\ref{sec:kmeans}).

Bisecting K-Means (see section~\ref{sec:kmeans-ext}) was implemented on top of scikit-learn: each time a data set is clustered
with k-means with $k = 2$. if the subset being clustered is big, 
($N > 2000$) then mini-batch k-means is used, otherwise usual K-means

Scatter/Gather (see section~\ref{sec:kmeans-ext}) was implemented manually 
using scipy \cite{scipy} and numpy \cite{walt2011numpy} because 
scikit-learn's implementation of K-Means does not allow using user-defined distances. 


DBScan (section~\ref{sec:dbscan}) and SNN Clustering (section~\ref{sec:dbscan-ext})
algorithms were also implemented manually:
available DBScan implementations usually take distance measure rather than 
a similarity measure, but we want to use a similarity matrix, and not 
a distance matrix because converting would make the matrix dense, 
and we want to avoid that. Additionally, available implementation
do not use the structure of data: in text-like data clustering algorithms 
can be sped up by using an inverted index (section~\ref{sec:index})


LSA (section~\ref{sec:lsa}) is implemented via randomized SVD \cite{tropp2009finding},
The implementation is taken from scikit-learn \cite{scikit-learn} - 
method \verb|randomized_svd|.

Non-negative Matrix Factorization (section~\ref{sec:nmf}) is also taken from 
cikit-learn \cite{scikit-learn}, class \verb|NMF|.


To assess the quality of clustering we used wikipedia categories. 
It is difficult to extract category information from raw wikipedia text, therefore
we used DBPedia \cite{bizer2009dbpedia} for that. Additionally 
DBPedia provides a SKOS ontology about categories. 


\subsection{Building Hierarchy} 

AMS Mathematics Subject Classification (2010) \cite{ams2010msc}
Excluded all subcategories those code end with '99': they are 
usually 'Miscellaneous topics' or 'None of the above, but in this section'. 
top level categories 'General', 'History and biography', and 'Mathematics education'
were also excluded. 
Additionally we exclude the following: 

\begin{itemize}
\item Quantum theory $\to$ Axiomatics, foundations, philosophy
\item Quantum theory $\to$ Applications to specific physical systems
\item Quantum theory $\to$ Groups and algebras in quantum theory
\item Partial differential equations $\to$ Equations of mathematical physics and other areas of application
\item Statistics $\to$ Sufficiency and information
\item Functional analysis $\to$ Other (nonclassical) types of functional analysis
\item Functional analysis $\to$ Miscellaneous applications of functional analysis
\end{itemize}

So these categories do not interfere with PACS. 


APS Physics and Astronomy Classification Scheme (2010) \cite{aps2010pacs}

We remove the ``GENERAL'' top-level category. 
In PACS there are 3 levels of categories, but we merge all 3-rd level categories into 
2nd level.

ACM Classification Scheme \cite{rous2012acm} available as a SKOS \cite{miles2005skos} ontology 
at their website \cite{amc2012ccs}. The SKOS ontology graph was processed with RDFLib \cite{rdflib}

We keep the following top level categories:
``Hardware'', ``Computer systems organization'', ``Networks'', 
``Software and its engineering'', ``Theory of computation'', 
``Information systems'', ``Security and privacy'',
``Human-centered computing'', ``Computing methodologies''.

After obtaining the data and parsing, all categories, the hierarchies 
are merged into one and then we try to match the found namespaces 
with second-level category in the hierarchy. 

This is done by keywords matching: we extract all words from the 
category (this includes top level category name, subcategory name
and all sub-sub categories concatenated). From the cluster 
we also extract the category information.
Then we try to do keyword matching using cosine similarity 
between the cluster and each category. The cluster is assigned 
to the category with the best cosine. 

If the cosine score is low (below $0.2$) or there is only one 
keyword matched, then the cluster is assigned to the ``OTHERS'' 
category.


\subsection{Java Language Processing}

The same set of techniques can be applied to source code.
(todo: why!)
If a language 
is statically typed, like Java or Pascal, usually it is possible to know the
type of a variable from its declaration. Therefore we can see 
variables as identifiers and types as "definitions" (TODO: clearly state 
the difference between types and definitions).

To extract this information from some source code repository each file source
file can be processed to obtain its Abstract Syntax Tree, and then 
declaration information can be extracted from it. 


In this word we process Java source code using JavaParser \cite{javaparser} - 
a library for parsing java source code. Java was chosen because the variable 
types always have to be declared, unlike other languages where the type 
can be inferred by compilers. 

The following declarations are processed: fields of a class, method and 
constructor parameters, inner variable declarations inside 
methods and constructors. It processes both usual classes and inner classes. 

\textbf{Add example and the results}


In the experiments we process Apache Mahout source code \cite{mahout}. 

Describe the dataset
