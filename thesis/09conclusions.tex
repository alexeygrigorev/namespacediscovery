\section{Conclusions}

The results are super. 


\subsection{Future Work}




% While I was reading, it occurred to me that one document may have several 
% namespaces, but I'm not yet sure how to model it. Maybe by taking a small 
% context around each formula and treating it as a small document? Or I 
% should hold this thought and wait till we maybe obtain some results 
% from other simpler methods?

the work is done assuming that document imports only from one namespace
but it can import from several. can solve that by dividing the document
in parts (e.g. by paragraphs) and then applying the same analysis 
independently to each paragraph - instead of each document. 


Can use additional information from wiki articles. For example, can 
extract some keywords from the article and use it in clustering

Or interwiki pages.

Pages that describe certain namespaces may be quite interconnected.
There are link-based clustering methods e.g. Botafogo and Schneiderman 1991

Can extract wiki graph and use this for clustering .
There are hybrid approaches that use both usual textual representation  + links
\cite{oikonomakou2005review}


It can be interesting to apply these techniques to a larger dataset, for example, arXiv.

Other dim red techniques for LSA, e.g. Local NMF \cite{li2001learning}
There should also be randomized NMF that works faster. 


Try other clustering techniques: spectral clustering \cite{ng2002spectral}
other ways to embed identifiers like word2vec \cite{mikolov2013efficient} 
or GloVe \cite{pennington2014glove}


How to extend this method to situations when no additional information 
about document category is known. I.e. need to replace the notion of 
purity with some other objective for discovering namespaces and 
namespace-defining clusters




