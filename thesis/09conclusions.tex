\section{Conclusions}




The obtained results indicate that there are namespaces in mathematical notation
and it is possible to discover them automatically from a collection of documents. 
However, there are still ...

Minimize the number of false definitions, and improve the results of clustering. 
In the next section we discuss how this can be approached.



\subsection{Future Work}

Clustering ensembles - cluster results are quite different, so 
need a way of combining several results.

\subsubsection{Implementation and Other Algorithms}  \ \\

% Implementation and % Other algorithms

We observe that cluster algorithms that produce many clusters 
tend to have good performance. However they also tend to create related 
clusters from the same category and with same or similar identifiers and 
definitions. Therefore such results can be refined further and merged. 
This can be done, for example, by using the join operation from 
the Scatter/Gather algorithm \cite{cutting1992scatter}.

The hierarchical agglomerative clustering algorithms may produce good 
clusters, but their time complexity is prohibitive. For these algorithms 
we are usually interested in the nearest neighbors of a given data point,
and therefore we can use approximation algorithms for computing nearest 
neighbors such as Locality-Sensitive Hashing (LSH) \cite{leskovec2014mining}.
The LSH algorithms can be used for text clustering \cite{ravichandran2005randomized},
and therefore they should work well for identifiers. Additionally, 
LSH is also a dimensionality reduction technique, and 
we have observed that generally reducing dimensionality helps to obtain 
better clusters. 

In this work we use hard assignment clustering algorithms, which means, that a document 
can import only from one namespace. This assumption does not necessarily always hold true 
and we may model the fact that documents may import from several namespaces by 
using Fuzzy Clustering (or Soft Clustering) algorithms \cite{baraldi1999survey}.

In Latent Semantic Analysis other dimensionality reduction techniques
can be used, for example, Local Non-Negative Matrix Factorization \cite{li2001learning}.
There are also randomized Non-Negative Matrix Factorization that use
random projections \cite{wang2010efficient} \cite{damle2014random}
that potentially can give a speed up while not significantly losing
in performance. Another dimensionality reduction technique useful for
discovering semantics is Dynamic Auto-Encoders \cite{mirowski2010dynamic}.

Topic modeling techniques such as Latent Dirichlet Allocation
\cite{blei2003latent} can be quite useful for modeling namespaces. It can be 
seen as a ``soft clustering'' technique and it can naturally model the fact that 
a document may import from several namespaces.

Finally, we can try different approaches to clustering such as 
Spectral Clustering \cite{ng2002spectral} or Micro-Clustering \cite{uno2015micro}.


\subsubsection{Other Concepts}  \ \\
% Other concepts

In this work we assume that document can import only from one namespace,
but in reality is should be able to import from several namespaces. As discussed,
it can be modeled by Fuzzy Clustering. But it also can be achieved by 
dividing the document in parts (for example, by paragraphs)
and then treating each part as an independent document.

In this work we only use identifiers, extracted definitions and categories.
It is possible to take advantage of additional information from Wikipedia
articles. For example, extract some keywords from the articles
and use them to get a better cluster assignment.

The Wikipedia data set can be seen as a graph, where two articles have 
an edge if there is an interwiki link between them. Pages that describe 
certain namespaces may be quite interconnected, and using this idea 
it is possible to apply link-based clustering methods (such as ones 
described in \cite{botafogo1991identifying} and \cite{johnson1996adaptive}) to 
find namespace candidates. There are also hybrid approaches that
can use both textual representation and links \cite{oikonomakou2005review}.

Vector Space Model is not the only possible model to represent textual 
information as vectors. There are other ways to embed textual information
into vector spaces like word2vec \cite{mikolov2013efficient} or 
GloVe \cite{pennington2014glove}, and these methods may be useful 
for representing identifers and definitions as well.

Additionally, tensors may be a better way of representing 
identifier-definition pairs. For example, we can represent the data set 
as a 3-dimensional tensor indexed by documents, identifiers and definition. 
Tensor Factorization methods for revealing semantic information 
are an active area of research in NLP and linguistics \cite{anisimov2014semantic},
and we can also apply these methods to the namespace discovery problem. 




\subsubsection{Datasets}  \ \\
% Other Datasets


It can be interesting to apply these techniques to a larger dataset, for example, arXiv.
ArXMLiv.

There are Q\&A websites on the stack exchange network with formulae, such as 
mathematics, mathoverflow, cross validated, data science, theoretical computer science, 
physics, astronomy, economics and many others that have math, and there are data dumps available 
from these sites with all the questions and answers. 


% Unsolved Questions

\subsubsection{Unsolved Questions} \ \\

There are questions that are still not answered and we are not certain 
how they can be approached. 

The biggest question is how to extend this method to situations when
no additional information about document category is known. To solve
it, we need to replace the notion of purity with some other objective
for discovering namespace-defining clusters.

We depend on existing hierarchies, and they are not always complete and
there are mismatches. And when this technique is applied to some
other language, a different hierarchy is needed for this language. When
we applied it to Russian, we needed to find a different hierarchy.
There should be a way of building these hierarchies
automatically, without the need of external dataset.
Potentially it should be possible to use hierarchical clustering 
algorithms, but it may result in very deep and unnatural hierarchies. 

Also, a metric for evaluating the quality of a namespace is needed.
Now we assume that pure clusters are namespace-defining clusters. But the namespace
candidates should adhere to the namespace definition as much as possible,
and therefore a good is needed.